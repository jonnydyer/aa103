\documentclass[twocolumn]{memoir} %twocolumn
\usepackage[utf8]{inputenc}
\usepackage[english]{babel}
\usepackage[T1]{fontenc}
% Nicer default font (+ math font) than Computer Modern for most use cases
\usepackage{mathpazo}
\usepackage[labelfont=bf]{caption}
\usepackage{xcolor} % Allow colors to be defined
\usepackage{enumerate} % Needed for markdown enumerations to work
\usepackage[]{geometry} % Used to adjust the document margins
\usepackage{amsmath} % Equations
\usepackage{amssymb} % Equations
\usepackage{booktabs}  
\usepackage{longtable}
\usepackage{wrapfig}
\usepackage{dblfloatfix}
\usepackage{tikz} 
\usepackage{pgfplots}
\usepackage{pgf}
\usepackage{float}
\usepackage{mhchem}
\usepackage{tabularx}
\usepackage{makecell}
\usepackage{footnote}
\usepackage{cancel}
\usepackage{listings}
\usepackage[sort&compress]{natbib}
\usepackage[pdftex,colorlinks]{hyperref}
\usepackage[noabbrev, capitalize]{cleveref}

\providecommand{\tightlist}{%
  \setlength{\itemsep}{0pt}\setlength{\parskip}{0pt}
}

\renewcommand*\familydefault{\sfdefault}
\renewcommand\tabularxcolumn[1]{m{#1}}

\makeatletter
    \newcommand\reaction@[1]{\begin{equation}\ce{#1}\end{equation}}
    \newcommand\reaction@nonumber[1]%
        {\begin{equation*}\ce{#1}\end{equation*}}
    \newcommand\reaction{\@ifstar{\reaction@nonumber}{\reaction@}}
\makeatother

\chapterstyle{demo2}
\setlrmarginsandblock{0.75in}{0.75in}{*}
\setulmarginsandblock{1in}{*}{1}
\checkandfixthelayout 

\usepackage{graphicx}
\graphicspath{{./imgs/}}

\title{Electric Propulsion}
\author{Jonny Dyer}
 
\begin{document}

\chapter*{Chemical Combustion}

\section{Motivation}
Having now seen what it takes to move beyond Earth and around the Solar System, we 
may start to see the limits of the old, faithful chemical propulsion systems.  With 
the highest performance practical propellant system, \ce{LO2} and \ce{LH2}, we might
achieve a vacuum specific impulse of around 450 seconds.  Even if we are willing to tolerate
all that goes along with dealing with deep cryogens for in-space propulsion, this performance
is still severly limiting from a mission mass perspective as can be seen in \cref{table:delta_V}.

\begin{table}[H]
\centering
    \begin{tabularx}{\columnwidth}{>{\raggedright\arraybackslash}X>{\raggedright\arraybackslash}c>{\raggedright\arraybackslash}c}
\toprule
        \textbf{Mission} & $\Delta V$ \textbf{Required} & $\frac{m_L}{m_i}$\\
\midrule
        LEO to GEO & 4 km/sec &  0.32 \\
        LEO to Lunar Orbit & 4 km/sec & 0.32 \\
        LEO to Low Mars Orbit & 6.6 km/sec & 0.14\\
        LEO to Moon Surface & 6-6.5 km/sec &  0.14 \\
        LEO to Mars Surface & 10.7 km/sec\footnotemark & 0.002\\
\bottomrule
\end{tabularx}
    \caption{Approximate mission $\Delta V$ for extra-LEO missions.  
    Assumes LOX+LH2 propellant, structural mass fraction of 9\% }    
\label{table:delta_V}        
\end{table}
\footnotetext{Since Mars has an atmosphere, aerobraking can be used, reducing this number somewhat}

This is probably the biggest hurdle to a manned Mars mission - even though it is possible to
achieve the required $\Delta V$ with chemical propellants, the payload delivered to surface
is a tiny fraction of the mass we put in LEO to start (0.2\%).  Furthermore, this is for the
minimum energy trajectory which takes nearly 200 days even with the best alignment of Earth and
Mars. For these travel durations, many other things become difficult - consumables add to 
mission mass and astronaut radiation exposure becomes a real concern.

For these reasons, many believe that moving about the solar system more routinely will require
different methods.  The method we will discuss here is Electric Propulsion.

\section{What is Electric Propulsion?}
Electric propulsion is a pretty broad term and essentially means any propulsion technology that
relies on electrical energy as the primary means to accelerate its exhaust gasses.  Given the
arguments in favor of chemical energy that we discussed in the combustion lectures, why would
anyone do such a thing?\footnote{Remember that batteries have 25$\times$ lower energy density,
4 order of magnitude lower power-density and are much, much better than in-space sources
like solar!}  The answer is that while dense, chemical combustion \emph{couples} the energy
source with the working fluid and thus the amount of energy we have availble is strictly
limited by how much propellant we bring along.  Propellant means mass and then we're up against
the tyranny of the rocket equation.  

Given
\begin{equation*}
    \Delta V = C \ln\frac{m_i}{m_f}
\end{equation*}
%
and
\begin{equation*}
    C^2 = \frac{\eta P_{in}}{\dot{m}} = \frac{\eta E_p}{m_p}
\end{equation*}
%
it follows that for \emph{any rocket system}
%
\begin{equation}
    \Delta V \sim \sqrt{\frac{\eta P_{in}}{\dot{m}}} \sim \sqrt{\frac{\eta E_p}{m_p}}.
    \label{eq:delV_P_dotm}
\end{equation}
%
In the case of chemical, gasdynamic rockets $P$ and $\dot{m}$ are related
%
\begin{equation*}
    P_{in} = \dot{m}\Delta H_{rxn}
\end{equation*}
%
and $\eta$ is a function of a number of things including $M_w$, $\gamma$ and system pressures
as we discussed in the lecture on rocket performance.  And so for chemical propulsion we arrive at
\begin{equation}
    \Delta V_{chem} \sim \sqrt{\Delta H_{rxn} \eta_{th}(\gamma, M_w, \ldots)}
\end{equation}

Note here that \emph{we really don't care what the input power is} because it is directly related
to other, more fundamental aspects of the chemistry and combustion.

If the energy source is independent of the working fluid (going back to \cref{eq:delV_P_dotm}), we can scale them separately and
perhaps achieve a more optimal result.  For instance solar, while not dense in power, is
essentially unlimited in total mission energy.  And nuclear fission may offer 3-4 $\times$
the energy density of the best chemical propellants.

Secondly, but separating the energy source from the working fluid we somewhat remove the
shackles of equillibrium gas-dynamics and the second law of thermodynamics.  Remember that
in chemical propulsion we throw as much as half or more of our energy away as unavailable 
thermal energy.  And that the achievable exhaust velocity is strictly tied to the molecular
weight of the propellants we choose.  Neither of these are necessarily true if we separate
working fluid from energy.

In general we find that electric propulsion comes in three basic forms:
\begin{itemize}
    \item \textbf{Electrothermal} - Gasses are heated electrically and then expanded through
        a gasdynamic nozzle, much like chemical propulsion.
    \item \textbf{Electrostatic} - Electrostatic body forces (Coulomb forces) accelerate charged
        ions to high velocities
    \item \textbf{Electromagnetic} - Ionized gasses accelerated by the interaction of external
        magnetic fields with the induced fields associated with plasma currents
\end{itemize}

At first blush electrothermal propulsion doesn't sound like much improvement over chemical.  However,
much higher temperatures may be achieved in an electrothermal thruster due to our ability to adjust
how much energy / mass we put in the working fluid.  Furthermore we may chose a working fluid with low
molecular weight (say hydrogen) without concern for where we get the chemical energy to go with it.
Finally in the case of arcjets (which will be discussed shortly), the heated gasses may be highly
anisotropic and non-equillibrium and thus not limited by second-law considerations.

Electrothermal systems are relatively straightforward and were developed earlier than electrostatic and
electromagnetic.  However the latter two have higher performance potential and have matured substantially
over the last 30 years.  The future is definitely in these two technologies.

\section{System-level performance considerations}

Given the propulsive efficiency defined as

\begin{equation*}
    \eta_p = \frac{\frac{1}{2}\dot{m}C^2}{P_{in}}
\end{equation*}
%
for thrusters with negligible exhaust pressure so that $C = U_e$.

Remembering
\begin{equation*}
    T = \dot{m}C
\end{equation*}
%
we arrive at
\begin{equation}
    T = \frac{2 \eta_p P_{in}}{C}
\end{equation}
%
Thus for any thruster we note that there is a relationship between thrust,
propulsive efficiency, input power and exhaust velocity (or specific impulse):

\begin{equation}
    \frac{T}{P_{in}} = \frac{2 \eta_p}{C} = \frac{2 \eta_p}{g_0 I_{sp}}
\end{equation}
%
For electric thrusters, the input power is delivered from an external source.
The source itself and the associated power conversion or conditioning systems
all have mass and in general that mass scales directly with power.

\begin{equation*}
    P_{in} = \alpha m_{pp}
\end{equation*}
%
where $m_{pp}$ this the mass of the power systems and other masses that scale 
with with power.  $\alpha$ is the system specific power density in W/kg.

If we assume a constant mass flowrate system

\begin{equation*}
    \dot{m}_p = \frac{m_p}{t_b}
\end{equation*}

and combining the above we arrive at

\begin{equation}
    m_{pp} = \frac{m_p C^2}{2 \alpha \eta_p t_b}
\end{equation}

This leads us to the observation that there is no free lunch - if we want a 
higher specific impulse we must either accept a hit on thrust or accept a dry
mass penalty in our system.  Since dry mass directly impacts $\Delta V$ through
the rocket equation, it is not obvious that this is a good thing.  We will also
see later that in real mission analysis, very small thrusts have other penalties.

Noting that the total initial system mass $m_i$ is
%
\begin{equation*}
    m_i = m_l + m_p + m_{pp} = m_l + m_p \left[1 + \frac{C^2}{2 \alpha \eta_p t_b}\right].
\end{equation*}
%
We can now write the rocket equation

\begin{equation*}
    \Delta V = C\ln{\left(\frac{ m_l + m_p \left[1 + \frac{C^2}{2 \alpha \eta_p t_b}\right]}
    { m_l + m_p \left[\frac{C^2}{2 \alpha \eta_p t_b}\right]}\right)}
\end{equation*}
%
Solving for the payload fraction

\begin{equation}
    \frac{m_l}{m_i} = \frac{1 - \left[e^{\Delta V / C} - 1\right]C^2/\left(2 \alpha \eta_p t_b\right)}{e^{\Delta V/C}}
\end{equation}
The grouping $2 \alpha \eta_p t_b$ is clearly an important one.  Considering units, we can define this as the square of a charcateristic speed such that

\begin{equation}
    v_c = \sqrt{2 \alpha \eta_p t_b}
\end{equation}
%
$v_c$ can be interpreted as the speed the dry mass of the powerplant would achieve if its power output were fully converted to kinetic energy of its mass, $m_{pp}$.

\section{Electrostatic Propulsion}

\begin{equation}
    \mathbf{f} = q\mathbf{E} = \frac{q^2}{4 \pi \epsilon_0 \| \mathbf{r}\|^2 }\mathbf{r}
\end{equation}

Richardson's Law:

\begin{equation}
    J = (1 - \tilde{r})A_0T^2 \exp{\left(\frac{-\phi}{kT}\right)}
\end{equation}

Child-Langmuir Law

\begin{equation}
    J = \frac{I}{A} = \frac{4 \epsilon_0}{9}\sqrt{\frac{2 q}{m_i}} \frac{V_T^{3/2}}{d^2}
\end{equation}

\begin{equation}
    \frac{F}{A} = \frac{8}{9}\epsilon_0 \left(\frac{V_T}{d}\right)^2
\end{equation}


\begin{equation}
    V_T = \frac{m_i C^2}{2 e}
\end{equation}

\bibliographystyle{unsrt}
\bibliography{refs}
\end{document}
