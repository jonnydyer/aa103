
% Default to the notebook output style

    


% Inherit from the specified cell style.




    
\documentclass[11pt]{article}

    
    
    \usepackage[T1]{fontenc}
    % Nicer default font (+ math font) than Computer Modern for most use cases
    \usepackage{mathpazo}

    % Basic figure setup, for now with no caption control since it's done
    % automatically by Pandoc (which extracts ![](path) syntax from Markdown).
    \usepackage{graphicx}
    % We will generate all images so they have a width \maxwidth. This means
    % that they will get their normal width if they fit onto the page, but
    % are scaled down if they would overflow the margins.
    \makeatletter
    \def\maxwidth{\ifdim\Gin@nat@width>\linewidth\linewidth
    \else\Gin@nat@width\fi}
    \makeatother
    \let\Oldincludegraphics\includegraphics
    % Set max figure width to be 80% of text width, for now hardcoded.
    \renewcommand{\includegraphics}[1]{\Oldincludegraphics[width=.8\maxwidth]{#1}}
    % Ensure that by default, figures have no caption (until we provide a
    % proper Figure object with a Caption API and a way to capture that
    % in the conversion process - todo).
    \usepackage{caption}
    \DeclareCaptionLabelFormat{nolabel}{}
    \captionsetup{labelformat=nolabel}

    \usepackage{adjustbox} % Used to constrain images to a maximum size 
    \usepackage{xcolor} % Allow colors to be defined
    \usepackage{enumerate} % Needed for markdown enumerations to work
    \usepackage{geometry} % Used to adjust the document margins
    \usepackage{amsmath} % Equations
    \usepackage{amssymb} % Equations
    \usepackage{textcomp} % defines textquotesingle
    % Hack from http://tex.stackexchange.com/a/47451/13684:
    \AtBeginDocument{%
        \def\PYZsq{\textquotesingle}% Upright quotes in Pygmentized code
    }
    \usepackage{upquote} % Upright quotes for verbatim code
    \usepackage{eurosym} % defines \euro
    \usepackage[mathletters]{ucs} % Extended unicode (utf-8) support
    \usepackage[utf8x]{inputenc} % Allow utf-8 characters in the tex document
    \usepackage{fancyvrb} % verbatim replacement that allows latex
    \usepackage{grffile} % extends the file name processing of package graphics 
                         % to support a larger range 
    % The hyperref package gives us a pdf with properly built
    % internal navigation ('pdf bookmarks' for the table of contents,
    % internal cross-reference links, web links for URLs, etc.)
    \usepackage{hyperref}
    \usepackage{longtable} % longtable support required by pandoc >1.10
    \usepackage{booktabs}  % table support for pandoc > 1.12.2
    \usepackage[inline]{enumitem} % IRkernel/repr support (it uses the enumerate* environment)
    \usepackage[normalem]{ulem} % ulem is needed to support strikethroughs (\sout)
                                % normalem makes italics be italics, not underlines
    \usepackage{mathrsfs}
    

    
    
    % Colors for the hyperref package
    \definecolor{urlcolor}{rgb}{0,.145,.698}
    \definecolor{linkcolor}{rgb}{.71,0.21,0.01}
    \definecolor{citecolor}{rgb}{.12,.54,.11}

    % ANSI colors
    \definecolor{ansi-black}{HTML}{3E424D}
    \definecolor{ansi-black-intense}{HTML}{282C36}
    \definecolor{ansi-red}{HTML}{E75C58}
    \definecolor{ansi-red-intense}{HTML}{B22B31}
    \definecolor{ansi-green}{HTML}{00A250}
    \definecolor{ansi-green-intense}{HTML}{007427}
    \definecolor{ansi-yellow}{HTML}{DDB62B}
    \definecolor{ansi-yellow-intense}{HTML}{B27D12}
    \definecolor{ansi-blue}{HTML}{208FFB}
    \definecolor{ansi-blue-intense}{HTML}{0065CA}
    \definecolor{ansi-magenta}{HTML}{D160C4}
    \definecolor{ansi-magenta-intense}{HTML}{A03196}
    \definecolor{ansi-cyan}{HTML}{60C6C8}
    \definecolor{ansi-cyan-intense}{HTML}{258F8F}
    \definecolor{ansi-white}{HTML}{C5C1B4}
    \definecolor{ansi-white-intense}{HTML}{A1A6B2}
    \definecolor{ansi-default-inverse-fg}{HTML}{FFFFFF}
    \definecolor{ansi-default-inverse-bg}{HTML}{000000}

    % commands and environments needed by pandoc snippets
    % extracted from the output of `pandoc -s`
    \providecommand{\tightlist}{%
      \setlength{\itemsep}{0pt}\setlength{\parskip}{0pt}}
    \DefineVerbatimEnvironment{Highlighting}{Verbatim}{commandchars=\\\{\}}
    % Add ',fontsize=\small' for more characters per line
    \newenvironment{Shaded}{}{}
    \newcommand{\KeywordTok}[1]{\textcolor[rgb]{0.00,0.44,0.13}{\textbf{{#1}}}}
    \newcommand{\DataTypeTok}[1]{\textcolor[rgb]{0.56,0.13,0.00}{{#1}}}
    \newcommand{\DecValTok}[1]{\textcolor[rgb]{0.25,0.63,0.44}{{#1}}}
    \newcommand{\BaseNTok}[1]{\textcolor[rgb]{0.25,0.63,0.44}{{#1}}}
    \newcommand{\FloatTok}[1]{\textcolor[rgb]{0.25,0.63,0.44}{{#1}}}
    \newcommand{\CharTok}[1]{\textcolor[rgb]{0.25,0.44,0.63}{{#1}}}
    \newcommand{\StringTok}[1]{\textcolor[rgb]{0.25,0.44,0.63}{{#1}}}
    \newcommand{\CommentTok}[1]{\textcolor[rgb]{0.38,0.63,0.69}{\textit{{#1}}}}
    \newcommand{\OtherTok}[1]{\textcolor[rgb]{0.00,0.44,0.13}{{#1}}}
    \newcommand{\AlertTok}[1]{\textcolor[rgb]{1.00,0.00,0.00}{\textbf{{#1}}}}
    \newcommand{\FunctionTok}[1]{\textcolor[rgb]{0.02,0.16,0.49}{{#1}}}
    \newcommand{\RegionMarkerTok}[1]{{#1}}
    \newcommand{\ErrorTok}[1]{\textcolor[rgb]{1.00,0.00,0.00}{\textbf{{#1}}}}
    \newcommand{\NormalTok}[1]{{#1}}
    
    % Additional commands for more recent versions of Pandoc
    \newcommand{\ConstantTok}[1]{\textcolor[rgb]{0.53,0.00,0.00}{{#1}}}
    \newcommand{\SpecialCharTok}[1]{\textcolor[rgb]{0.25,0.44,0.63}{{#1}}}
    \newcommand{\VerbatimStringTok}[1]{\textcolor[rgb]{0.25,0.44,0.63}{{#1}}}
    \newcommand{\SpecialStringTok}[1]{\textcolor[rgb]{0.73,0.40,0.53}{{#1}}}
    \newcommand{\ImportTok}[1]{{#1}}
    \newcommand{\DocumentationTok}[1]{\textcolor[rgb]{0.73,0.13,0.13}{\textit{{#1}}}}
    \newcommand{\AnnotationTok}[1]{\textcolor[rgb]{0.38,0.63,0.69}{\textbf{\textit{{#1}}}}}
    \newcommand{\CommentVarTok}[1]{\textcolor[rgb]{0.38,0.63,0.69}{\textbf{\textit{{#1}}}}}
    \newcommand{\VariableTok}[1]{\textcolor[rgb]{0.10,0.09,0.49}{{#1}}}
    \newcommand{\ControlFlowTok}[1]{\textcolor[rgb]{0.00,0.44,0.13}{\textbf{{#1}}}}
    \newcommand{\OperatorTok}[1]{\textcolor[rgb]{0.40,0.40,0.40}{{#1}}}
    \newcommand{\BuiltInTok}[1]{{#1}}
    \newcommand{\ExtensionTok}[1]{{#1}}
    \newcommand{\PreprocessorTok}[1]{\textcolor[rgb]{0.74,0.48,0.00}{{#1}}}
    \newcommand{\AttributeTok}[1]{\textcolor[rgb]{0.49,0.56,0.16}{{#1}}}
    \newcommand{\InformationTok}[1]{\textcolor[rgb]{0.38,0.63,0.69}{\textbf{\textit{{#1}}}}}
    \newcommand{\WarningTok}[1]{\textcolor[rgb]{0.38,0.63,0.69}{\textbf{\textit{{#1}}}}}
    
    
    % Define a nice break command that doesn't care if a line doesn't already
    % exist.
    \def\br{\hspace*{\fill} \\* }
    % Math Jax compatibility definitions
    \def\gt{>}
    \def\lt{<}
    \let\Oldtex\TeX
    \let\Oldlatex\LaTeX
    \renewcommand{\TeX}{\textrm{\Oldtex}}
    \renewcommand{\LaTeX}{\textrm{\Oldlatex}}
    % Document parameters
    % Document title
    \title{SpecificImpulseEfficiency}
    
    
    
    
    

    % Pygments definitions
    
\makeatletter
\def\PY@reset{\let\PY@it=\relax \let\PY@bf=\relax%
    \let\PY@ul=\relax \let\PY@tc=\relax%
    \let\PY@bc=\relax \let\PY@ff=\relax}
\def\PY@tok#1{\csname PY@tok@#1\endcsname}
\def\PY@toks#1+{\ifx\relax#1\empty\else%
    \PY@tok{#1}\expandafter\PY@toks\fi}
\def\PY@do#1{\PY@bc{\PY@tc{\PY@ul{%
    \PY@it{\PY@bf{\PY@ff{#1}}}}}}}
\def\PY#1#2{\PY@reset\PY@toks#1+\relax+\PY@do{#2}}

\expandafter\def\csname PY@tok@gd\endcsname{\def\PY@tc##1{\textcolor[rgb]{0.63,0.00,0.00}{##1}}}
\expandafter\def\csname PY@tok@gu\endcsname{\let\PY@bf=\textbf\def\PY@tc##1{\textcolor[rgb]{0.50,0.00,0.50}{##1}}}
\expandafter\def\csname PY@tok@gt\endcsname{\def\PY@tc##1{\textcolor[rgb]{0.00,0.27,0.87}{##1}}}
\expandafter\def\csname PY@tok@gs\endcsname{\let\PY@bf=\textbf}
\expandafter\def\csname PY@tok@gr\endcsname{\def\PY@tc##1{\textcolor[rgb]{1.00,0.00,0.00}{##1}}}
\expandafter\def\csname PY@tok@cm\endcsname{\let\PY@it=\textit\def\PY@tc##1{\textcolor[rgb]{0.25,0.50,0.50}{##1}}}
\expandafter\def\csname PY@tok@vg\endcsname{\def\PY@tc##1{\textcolor[rgb]{0.10,0.09,0.49}{##1}}}
\expandafter\def\csname PY@tok@vi\endcsname{\def\PY@tc##1{\textcolor[rgb]{0.10,0.09,0.49}{##1}}}
\expandafter\def\csname PY@tok@vm\endcsname{\def\PY@tc##1{\textcolor[rgb]{0.10,0.09,0.49}{##1}}}
\expandafter\def\csname PY@tok@mh\endcsname{\def\PY@tc##1{\textcolor[rgb]{0.40,0.40,0.40}{##1}}}
\expandafter\def\csname PY@tok@cs\endcsname{\let\PY@it=\textit\def\PY@tc##1{\textcolor[rgb]{0.25,0.50,0.50}{##1}}}
\expandafter\def\csname PY@tok@ge\endcsname{\let\PY@it=\textit}
\expandafter\def\csname PY@tok@vc\endcsname{\def\PY@tc##1{\textcolor[rgb]{0.10,0.09,0.49}{##1}}}
\expandafter\def\csname PY@tok@il\endcsname{\def\PY@tc##1{\textcolor[rgb]{0.40,0.40,0.40}{##1}}}
\expandafter\def\csname PY@tok@go\endcsname{\def\PY@tc##1{\textcolor[rgb]{0.53,0.53,0.53}{##1}}}
\expandafter\def\csname PY@tok@cp\endcsname{\def\PY@tc##1{\textcolor[rgb]{0.74,0.48,0.00}{##1}}}
\expandafter\def\csname PY@tok@gi\endcsname{\def\PY@tc##1{\textcolor[rgb]{0.00,0.63,0.00}{##1}}}
\expandafter\def\csname PY@tok@gh\endcsname{\let\PY@bf=\textbf\def\PY@tc##1{\textcolor[rgb]{0.00,0.00,0.50}{##1}}}
\expandafter\def\csname PY@tok@ni\endcsname{\let\PY@bf=\textbf\def\PY@tc##1{\textcolor[rgb]{0.60,0.60,0.60}{##1}}}
\expandafter\def\csname PY@tok@nl\endcsname{\def\PY@tc##1{\textcolor[rgb]{0.63,0.63,0.00}{##1}}}
\expandafter\def\csname PY@tok@nn\endcsname{\let\PY@bf=\textbf\def\PY@tc##1{\textcolor[rgb]{0.00,0.00,1.00}{##1}}}
\expandafter\def\csname PY@tok@no\endcsname{\def\PY@tc##1{\textcolor[rgb]{0.53,0.00,0.00}{##1}}}
\expandafter\def\csname PY@tok@na\endcsname{\def\PY@tc##1{\textcolor[rgb]{0.49,0.56,0.16}{##1}}}
\expandafter\def\csname PY@tok@nb\endcsname{\def\PY@tc##1{\textcolor[rgb]{0.00,0.50,0.00}{##1}}}
\expandafter\def\csname PY@tok@nc\endcsname{\let\PY@bf=\textbf\def\PY@tc##1{\textcolor[rgb]{0.00,0.00,1.00}{##1}}}
\expandafter\def\csname PY@tok@nd\endcsname{\def\PY@tc##1{\textcolor[rgb]{0.67,0.13,1.00}{##1}}}
\expandafter\def\csname PY@tok@ne\endcsname{\let\PY@bf=\textbf\def\PY@tc##1{\textcolor[rgb]{0.82,0.25,0.23}{##1}}}
\expandafter\def\csname PY@tok@nf\endcsname{\def\PY@tc##1{\textcolor[rgb]{0.00,0.00,1.00}{##1}}}
\expandafter\def\csname PY@tok@si\endcsname{\let\PY@bf=\textbf\def\PY@tc##1{\textcolor[rgb]{0.73,0.40,0.53}{##1}}}
\expandafter\def\csname PY@tok@s2\endcsname{\def\PY@tc##1{\textcolor[rgb]{0.73,0.13,0.13}{##1}}}
\expandafter\def\csname PY@tok@nt\endcsname{\let\PY@bf=\textbf\def\PY@tc##1{\textcolor[rgb]{0.00,0.50,0.00}{##1}}}
\expandafter\def\csname PY@tok@nv\endcsname{\def\PY@tc##1{\textcolor[rgb]{0.10,0.09,0.49}{##1}}}
\expandafter\def\csname PY@tok@s1\endcsname{\def\PY@tc##1{\textcolor[rgb]{0.73,0.13,0.13}{##1}}}
\expandafter\def\csname PY@tok@dl\endcsname{\def\PY@tc##1{\textcolor[rgb]{0.73,0.13,0.13}{##1}}}
\expandafter\def\csname PY@tok@ch\endcsname{\let\PY@it=\textit\def\PY@tc##1{\textcolor[rgb]{0.25,0.50,0.50}{##1}}}
\expandafter\def\csname PY@tok@m\endcsname{\def\PY@tc##1{\textcolor[rgb]{0.40,0.40,0.40}{##1}}}
\expandafter\def\csname PY@tok@gp\endcsname{\let\PY@bf=\textbf\def\PY@tc##1{\textcolor[rgb]{0.00,0.00,0.50}{##1}}}
\expandafter\def\csname PY@tok@sh\endcsname{\def\PY@tc##1{\textcolor[rgb]{0.73,0.13,0.13}{##1}}}
\expandafter\def\csname PY@tok@ow\endcsname{\let\PY@bf=\textbf\def\PY@tc##1{\textcolor[rgb]{0.67,0.13,1.00}{##1}}}
\expandafter\def\csname PY@tok@sx\endcsname{\def\PY@tc##1{\textcolor[rgb]{0.00,0.50,0.00}{##1}}}
\expandafter\def\csname PY@tok@bp\endcsname{\def\PY@tc##1{\textcolor[rgb]{0.00,0.50,0.00}{##1}}}
\expandafter\def\csname PY@tok@c1\endcsname{\let\PY@it=\textit\def\PY@tc##1{\textcolor[rgb]{0.25,0.50,0.50}{##1}}}
\expandafter\def\csname PY@tok@fm\endcsname{\def\PY@tc##1{\textcolor[rgb]{0.00,0.00,1.00}{##1}}}
\expandafter\def\csname PY@tok@o\endcsname{\def\PY@tc##1{\textcolor[rgb]{0.40,0.40,0.40}{##1}}}
\expandafter\def\csname PY@tok@kc\endcsname{\let\PY@bf=\textbf\def\PY@tc##1{\textcolor[rgb]{0.00,0.50,0.00}{##1}}}
\expandafter\def\csname PY@tok@c\endcsname{\let\PY@it=\textit\def\PY@tc##1{\textcolor[rgb]{0.25,0.50,0.50}{##1}}}
\expandafter\def\csname PY@tok@mf\endcsname{\def\PY@tc##1{\textcolor[rgb]{0.40,0.40,0.40}{##1}}}
\expandafter\def\csname PY@tok@err\endcsname{\def\PY@bc##1{\setlength{\fboxsep}{0pt}\fcolorbox[rgb]{1.00,0.00,0.00}{1,1,1}{\strut ##1}}}
\expandafter\def\csname PY@tok@mb\endcsname{\def\PY@tc##1{\textcolor[rgb]{0.40,0.40,0.40}{##1}}}
\expandafter\def\csname PY@tok@ss\endcsname{\def\PY@tc##1{\textcolor[rgb]{0.10,0.09,0.49}{##1}}}
\expandafter\def\csname PY@tok@sr\endcsname{\def\PY@tc##1{\textcolor[rgb]{0.73,0.40,0.53}{##1}}}
\expandafter\def\csname PY@tok@mo\endcsname{\def\PY@tc##1{\textcolor[rgb]{0.40,0.40,0.40}{##1}}}
\expandafter\def\csname PY@tok@kd\endcsname{\let\PY@bf=\textbf\def\PY@tc##1{\textcolor[rgb]{0.00,0.50,0.00}{##1}}}
\expandafter\def\csname PY@tok@mi\endcsname{\def\PY@tc##1{\textcolor[rgb]{0.40,0.40,0.40}{##1}}}
\expandafter\def\csname PY@tok@kn\endcsname{\let\PY@bf=\textbf\def\PY@tc##1{\textcolor[rgb]{0.00,0.50,0.00}{##1}}}
\expandafter\def\csname PY@tok@cpf\endcsname{\let\PY@it=\textit\def\PY@tc##1{\textcolor[rgb]{0.25,0.50,0.50}{##1}}}
\expandafter\def\csname PY@tok@kr\endcsname{\let\PY@bf=\textbf\def\PY@tc##1{\textcolor[rgb]{0.00,0.50,0.00}{##1}}}
\expandafter\def\csname PY@tok@s\endcsname{\def\PY@tc##1{\textcolor[rgb]{0.73,0.13,0.13}{##1}}}
\expandafter\def\csname PY@tok@kp\endcsname{\def\PY@tc##1{\textcolor[rgb]{0.00,0.50,0.00}{##1}}}
\expandafter\def\csname PY@tok@w\endcsname{\def\PY@tc##1{\textcolor[rgb]{0.73,0.73,0.73}{##1}}}
\expandafter\def\csname PY@tok@kt\endcsname{\def\PY@tc##1{\textcolor[rgb]{0.69,0.00,0.25}{##1}}}
\expandafter\def\csname PY@tok@sc\endcsname{\def\PY@tc##1{\textcolor[rgb]{0.73,0.13,0.13}{##1}}}
\expandafter\def\csname PY@tok@sb\endcsname{\def\PY@tc##1{\textcolor[rgb]{0.73,0.13,0.13}{##1}}}
\expandafter\def\csname PY@tok@sa\endcsname{\def\PY@tc##1{\textcolor[rgb]{0.73,0.13,0.13}{##1}}}
\expandafter\def\csname PY@tok@k\endcsname{\let\PY@bf=\textbf\def\PY@tc##1{\textcolor[rgb]{0.00,0.50,0.00}{##1}}}
\expandafter\def\csname PY@tok@se\endcsname{\let\PY@bf=\textbf\def\PY@tc##1{\textcolor[rgb]{0.73,0.40,0.13}{##1}}}
\expandafter\def\csname PY@tok@sd\endcsname{\let\PY@it=\textit\def\PY@tc##1{\textcolor[rgb]{0.73,0.13,0.13}{##1}}}

\def\PYZbs{\char`\\}
\def\PYZus{\char`\_}
\def\PYZob{\char`\{}
\def\PYZcb{\char`\}}
\def\PYZca{\char`\^}
\def\PYZam{\char`\&}
\def\PYZlt{\char`\<}
\def\PYZgt{\char`\>}
\def\PYZsh{\char`\#}
\def\PYZpc{\char`\%}
\def\PYZdl{\char`\$}
\def\PYZhy{\char`\-}
\def\PYZsq{\char`\'}
\def\PYZdq{\char`\"}
\def\PYZti{\char`\~}
% for compatibility with earlier versions
\def\PYZat{@}
\def\PYZlb{[}
\def\PYZrb{]}
\makeatother


    % Exact colors from NB
    \definecolor{incolor}{rgb}{0.0, 0.0, 0.5}
    \definecolor{outcolor}{rgb}{0.545, 0.0, 0.0}



    
    % Prevent overflowing lines due to hard-to-break entities
    \sloppy 
    % Setup hyperref package
    \hypersetup{
      breaklinks=true,  % so long urls are correctly broken across lines
      colorlinks=true,
      urlcolor=urlcolor,
      linkcolor=linkcolor,
      citecolor=citecolor,
      }
    % Slightly bigger margins than the latex defaults
    
    \geometry{verbose,tmargin=1in,bmargin=1in,lmargin=1in,rmargin=1in}
    
    

    \begin{document}
    
    
    \maketitle
    
    

    
    \begin{Verbatim}[commandchars=\\\{\}]
{\color{incolor}In [{\color{incolor}1}]:} \PY{o}{\PYZpc{}}\PY{k}{pylab} inline
        \PY{o}{\PYZpc{}}\PY{k}{config} InlineBackend.figure\PYZus{}format = \PYZsq{}retina\PYZsq{}
        \PY{n}{rc}\PY{p}{(}\PY{l+s+s1}{\PYZsq{}}\PY{l+s+s1}{figure}\PY{l+s+s1}{\PYZsq{}}\PY{p}{,} \PY{n}{figsize}\PY{o}{=}\PY{p}{(}\PY{l+m+mi}{4}\PY{p}{,}\PY{l+m+mf}{2.5}\PY{p}{)}\PY{p}{)}
        \PY{n}{rc}\PY{p}{(}\PY{l+s+s1}{\PYZsq{}}\PY{l+s+s1}{legend}\PY{l+s+s1}{\PYZsq{}}\PY{p}{,} \PY{n}{fontsize}\PY{o}{=}\PY{l+s+s1}{\PYZsq{}}\PY{l+s+s1}{small}\PY{l+s+s1}{\PYZsq{}}\PY{p}{)}
        \PY{n}{rc}\PY{p}{(}\PY{l+s+s1}{\PYZsq{}}\PY{l+s+s1}{font}\PY{l+s+s1}{\PYZsq{}}\PY{p}{,} \PY{n}{family}\PY{o}{=}\PY{l+s+s1}{\PYZsq{}}\PY{l+s+s1}{serif}\PY{l+s+s1}{\PYZsq{}}\PY{p}{)}
        \PY{c+c1}{\PYZsh{}rc(\PYZsq{}xtick\PYZsq{}, labelsize=\PYZsq{}small\PYZsq{})}
        \PY{c+c1}{\PYZsh{}rc(\PYZsq{}ytick\PYZsq{}, labelsize=\PYZsq{}small\PYZsq{})}
        
        \PY{k}{def} \PY{n+nf}{fix\PYZus{}fig}\PY{p}{(}\PY{n}{f}\PY{p}{)}\PY{p}{:}
            \PY{k}{for} \PY{n}{a} \PY{o+ow}{in} \PY{n}{f}\PY{o}{.}\PY{n}{get\PYZus{}axes}\PY{p}{(}\PY{p}{)}\PY{p}{:}
                \PY{n}{a}\PY{o}{.}\PY{n}{spines}\PY{p}{[}\PY{l+s+s1}{\PYZsq{}}\PY{l+s+s1}{right}\PY{l+s+s1}{\PYZsq{}}\PY{p}{]}\PY{o}{.}\PY{n}{set\PYZus{}visible}\PY{p}{(}\PY{n+nb+bp}{False}\PY{p}{)}
                \PY{n}{a}\PY{o}{.}\PY{n}{spines}\PY{p}{[}\PY{l+s+s1}{\PYZsq{}}\PY{l+s+s1}{top}\PY{l+s+s1}{\PYZsq{}}\PY{p}{]}\PY{o}{.}\PY{n}{set\PYZus{}visible}\PY{p}{(}\PY{n+nb+bp}{False}\PY{p}{)}
                \PY{c+c1}{\PYZsh{} Only show ticks on the left and bottom spines}
                \PY{c+c1}{\PYZsh{}a.yaxis.set\PYZus{}ticks\PYZus{}position(\PYZsq{}left\PYZsq{})}
                \PY{n}{a}\PY{o}{.}\PY{n}{xaxis}\PY{o}{.}\PY{n}{set\PYZus{}ticks\PYZus{}position}\PY{p}{(}\PY{l+s+s1}{\PYZsq{}}\PY{l+s+s1}{bottom}\PY{l+s+s1}{\PYZsq{}}\PY{p}{)}
            \PY{n}{f}\PY{o}{.}\PY{n}{set\PYZus{}tight\PYZus{}layout}\PY{p}{(}\PY{n+nb+bp}{True}\PY{p}{)}
\end{Verbatim}

    \begin{Verbatim}[commandchars=\\\{\}]
Populating the interactive namespace from numpy and matplotlib

    \end{Verbatim}

    \begin{Verbatim}[commandchars=\\\{\}]

Bad key "axes.color\_cycle" on line 240 in
/Users/jonny/Library/Mobile Documents/com\textasciitilde{}apple\textasciitilde{}CloudDocs/Documents/YHayPi/aa103/matplotlibrc.
You probably need to get an updated matplotlibrc file from
http://github.com/matplotlib/matplotlib/blob/master/matplotlibrc.template
or from the matplotlib source distribution

    \end{Verbatim}

    \section{Specific Impulse, SFC and
Efficiencies}\label{specific-impulse-sfc-and-efficiencies}

\subsubsection{Assumptions for topics already
covered}\label{assumptions-for-topics-already-covered}

\begin{itemize}
\tightlist
\item
  Equation for thrust -
  \(T = \dot{m}_{in}(U_e - U_0) + (P_e - P_0)A_e + \dot{m}_pU_e\)

  \begin{itemize}
  \tightlist
  \item
    Reduces to \(T = (P_e - P_0)A_e + \dot{m}_pU_e\) for rocket case
  \end{itemize}
\item
  Rocket equation - \(\Delta V = C \ln \frac{m_0}{m_f}\)
\item
  Isentropic flow through nozzles, critical flow, supersonic flow in de
  laval nozzle
\end{itemize}

\subsection{C and Isp}\label{c-and-isp}

Now that we have covered the basic mechanics of thrust and the
isentropic nozzle relations we will begin to show how these apply to
rockets by defining a set of convenient parameters.

Let's start by remembering that, for a rocket

\[T = (P_e - P_0)A_e + \dot{m}_pU_e\]

Our intuition tells us that a good parameter for the effectiveness of a
rocket is the quotient of thrust (what we want) with propellant mass
flowrate (what we have to pay):

\[\frac{T}{\dot{m}} = (P_e - P_0)\frac{Ae}{\dot{m}} + U_e\]

This parameter is called effective exhaust velocity,
\(C = \frac{T}{\dot{m}}\) and is one of the most important in rocketry.
It figures prominently in the Rocket Equation

\[\Delta V = C \ln \frac{M_0}{M_f}\]

which we will be derived in a couple lectures.

If we assume \(T\) and \(\dot{m}\) are completely independent (which we
will see is not always true), we can also interpret \(C\) as

\[C = \frac{\int T dt}{\int \dot{m} dt} = \frac{I}{M_p}\]

which is how much total impulse we get from a unit mass of propellant.
This interpretation points to a very closely related parameter called
Specific Impulse, \(I_{sp}\):

\[I_{sp} = \frac{C}{g_0} = \frac{T}{g_0\dot{m}}\]

Notice that \(I_{sp}\) is just \(C\) normalized by Earth's gravitational
acceleration with overall units of seconds. While at first a bit
non-sensical, \(I_{sp}\) makes sense when you consider your propellant
consumption as a weight flowrate rather than a mass-flowrate and so you
are dividing thrust (a force) by weight flowrate (force per time) to
arrive at a time. The etymology of \(I_{sp}\) is rooted in the fact that
for imperial units we typically work in lbm. rather than slugs.

And while weird at first, the units of seconds do have some intuitive
utility. For a rocket with \(I_{sp} = 100 \text{s}\) a unit mass, \(m\)
of propellant can generate \(T=g_0 m\) thrust for 100 seconds or
\(T = 100 g_0 m\) thrust for one second.

Specific impulse is popularly spoken of as the "gas mileage" for a
rocket cycle and this is fairly reasonable - it fundamentally indicates
how much bang for the buck you get. I'll jump the gun just a bit for the
sake of intuition and give some typical \(I_{sp}\) values for different
types of propulsion:

\begin{verbatim}
<th>Technology</th>
<th>Isp (s)</th>
<th>Exhaust Velocity (m/s)</th>
\end{verbatim}

\begin{verbatim}
<td>Reflective Photon Propulsion</td>
<td>$\infty$</td>
<td>$\infty$</td>
\end{verbatim}

\begin{verbatim}
<td>Electric Propulsion</td>
<td>1000 - 10,000</td>
<td>9,800 - 98,000</td>
\end{verbatim}

\begin{verbatim}
<td>Nuclear Thermal / Beamed Energy Propulsion</td>
<td>600 - 1,000</td>
<td>5,900 - 9,800</td>
\end{verbatim}

\begin{verbatim}
<td>Bipropellant Chemical Propulsion</td>
<td>200 - 500</td>
<td>2,000 - 4,900</td>
\end{verbatim}

\begin{verbatim}
<td>Monopropellant Chemical Propulsion</td>
<td>100 - 250</td>
<td>980 - 2,450</td>
\end{verbatim}

\begin{verbatim}
<td>Cold Gas Propulsion</td>
<td>10 - 120</td>
<td>100 - 1,150</td>
\end{verbatim}

I bet you're intrigued by the first row... not to worry this will be
covered later in the course.

This is all well and good, but all we have really done at this point is
some algebra. What we really are interested in as engineers is how do I
get the most gas mileage out of my rocket. And for that discussion,
we'll first define another couple useful parameters.

    \subsection{c*}\label{c}

No we will go back, for a moment, to choked compressible flow. With the
isentropic flow equations and the \(M=1\) choked condition, we can
derive (assuming constant \(C_p\), \(C_v\) and \(R\)):

\[\dot{m} = \rho_c a_c A_c  = \rho_t \left[ \frac{\rho_c}{\rho_t}\right]a_0 \left[ \frac{a_c}{a_t}\right]A_c = \frac{P_t A_c}{\sqrt{\frac{R T_t}{\gamma}}} \left[\frac{\gamma + 1}{2}\right]^{\frac{2(\gamma-1)}{\gamma+1}}\]

remembering that the \(t\) subscript denotes the total or stagnation
condition and the \(c\) subscript denotes choked (\(M=1\)) condition.
This is a very useful relationship as it allows us to compute the mass
flowrate through a choked nozzle as a function of only nozzle throat
area, \(A_c\), ideal gas properties and the stangation condition
temperature and pressure.

Let's define a new parameter, called characteristic velocity and denoted
\(c^{*}\) using this result:

\[c^* = \frac{P_t A_c}{\dot{m}} = \sqrt{\frac{R T_t}{\gamma}}\left[\frac{\gamma + 1}{2}\right]^{\frac{\gamma + 1}{2(\gamma - 1)}}\]

\(c^*\) is useful far beyond simple choked flow considerations because
it can be both \textbf{measured} and \textbf{computed}. And indeed the
equation above shows this directly - the LHS is a function relatively
measure-able variables pressure, area and mass flowrate. The RHS is a
function of intrinsic gas properties \(\gamma\), \(R\) and stagnation
temperature. \(c^*\) gives us the tools to calculate a parameter (RHS)
that we can go and easily measure in the lab (LHS).

Given that \(c^*\) depends on \(\gamma\), \(R\) and \(T_t\) it is
essentially a function only of the working fluids thermodynamic
properties and stagnation (chamber state). This is not 100\% true - when
we get to combustion we will see how a (weak) dependency on pressure
comes back into \(c^*\), but for conceptual purposes we should think of
\(c^*\) this way - as only a function of intrinsic thermodynamic
properties of our propellants.

And so finally we come to a very interesting result regarding the
functional dependence of \(c^*\):

\[c^* = f(T_t, M_w, \gamma)\]

\(R\) is the specific gas constant \(R = \frac{R_u}{M_w}\) which is
inversely dependent on gas molecular weight, \(M_w\). We can say that

\[c^* \propto \sqrt{\frac{T_t}{M_w}}\]

Moreover \(T_t\) is related to the gas internal stagnation enthalpy by
\(\Delta h_t = \int_{T_0}^{T_t} C_p dT\) and so we should see \(T_t\) as
a representation of the \textbf{energy content} of the rocket gasses.

The effect of \(\gamma\) on \(c^*\) is a bit more subtle. \(\gamma\) is
fundamentally related to the number of vibratory degrees of freedom a
molecule has. For monatomic systems (such as helium or atomic hydrogen)
\(\gamma\) assymptotes to an upper limit of 1.66. For most simple
diatoms (nitrogen, oxygen, hydrogen) it is around 1.4 and for larger
molecules or those with more complex bonding it is lower. For the
conditions we are interested in within a rocket, we wouldn't expect to
find \(\gamma\) much lower than 1.1.

The plot below shows the effect of \(\gamma\) on \(c^*\) which is fairly
limited compared with \(T_t\) and \(M_w\). And so if you take one thing
away from this discussion, remember that

\[c^* \propto \sqrt{\frac{\Delta h_t}{\overline{C_p} M_w}}\]

    \begin{Verbatim}[commandchars=\\\{\}]
{\color{incolor}In [{\color{incolor}2}]:} \PY{n}{gamma} \PY{o}{=} \PY{n}{np}\PY{o}{.}\PY{n}{linspace}\PY{p}{(}\PY{l+m+mf}{1.1}\PY{p}{,} \PY{l+m+mf}{1.66}\PY{p}{)}
        \PY{n}{cstar\PYZus{}gamm} \PY{o}{=} \PY{n}{sqrt}\PY{p}{(}\PY{l+m+mf}{1.} \PY{o}{/} \PY{n}{gamma}\PY{p}{)} \PY{o}{*} \PY{p}{(}\PY{p}{(}\PY{n}{gamma} \PY{o}{+} \PY{l+m+mf}{1.}\PY{p}{)}\PY{o}{/}\PY{l+m+mf}{2.}\PY{p}{)}\PY{o}{*}\PY{o}{*}\PY{p}{(}\PY{p}{(}\PY{n}{gamma} \PY{o}{+} \PY{l+m+mf}{1.}\PY{p}{)} \PY{o}{/} \PY{l+m+mf}{2.} \PY{o}{/} \PY{p}{(}\PY{n}{gamma} \PY{o}{\PYZhy{}} \PY{l+m+mf}{1.}\PY{p}{)}\PY{p}{)}
        \PY{n}{f} \PY{o}{=} \PY{n}{plt}\PY{o}{.}\PY{n}{figure}\PY{p}{(}\PY{p}{)}
        \PY{n}{plt}\PY{o}{.}\PY{n}{plot}\PY{p}{(}\PY{n}{gamma}\PY{p}{,} \PY{n}{cstar\PYZus{}gamm} \PY{o}{/} \PY{n}{cstar\PYZus{}gamm}\PY{p}{[}\PY{l+m+mi}{0}\PY{p}{]}\PY{p}{)}
        \PY{n}{plt}\PY{o}{.}\PY{n}{xlabel}\PY{p}{(}\PY{l+s+sa}{r}\PY{l+s+s1}{\PYZsq{}}\PY{l+s+s1}{\PYZdl{}}\PY{l+s+s1}{\PYZbs{}}\PY{l+s+s1}{gamma\PYZdl{}}\PY{l+s+s1}{\PYZsq{}}\PY{p}{)}
        \PY{n}{plt}\PY{o}{.}\PY{n}{ylabel}\PY{p}{(}\PY{l+s+sa}{r}\PY{l+s+s1}{\PYZsq{}}\PY{l+s+s1}{\PYZdl{}c\PYZca{}*\PYZdl{}}\PY{l+s+s1}{\PYZsq{}}\PY{p}{)}
        \PY{n}{fix\PYZus{}fig}\PY{p}{(}\PY{n}{f}\PY{p}{)}
        \PY{n}{f}\PY{o}{.}\PY{n}{savefig}\PY{p}{(}\PY{l+s+s1}{\PYZsq{}}\PY{l+s+s1}{imgs/cstar\PYZus{}gamma.pdf}\PY{l+s+s1}{\PYZsq{}}\PY{p}{)}
\end{Verbatim}

    \begin{Verbatim}[commandchars=\\\{\}]
/Users/jonny/anaconda/lib/python2.7/site-packages/matplotlib/figure.py:2299: UserWarning: This figure includes Axes that are not compatible with tight\_layout, so results might be incorrect.
  warnings.warn("This figure includes Axes that are not compatible "

    \end{Verbatim}

    \begin{center}
    \adjustimage{max size={0.9\linewidth}{0.9\paperheight}}{output_3_1.png}
    \end{center}
    { \hspace*{\fill} \\}
    
    \subsection{\texorpdfstring{\(C_f\)}{C\_f}}\label{c_f}

Ok, that's interesting, but for the purposes of rocket performance we
want \(C\) not \(c^*\). Take a look at the definition of \(c^*\) and it
looks a lot like our definition of effective velocity and specific
impulse above:

\[c^* = \frac{P_t A_c}{\dot{m}} \sim \frac{T}{\dot{m}} = C\]

and indeed \(c^*\) and \(C\) \textbf{are} closely related.

In order to see how, let's define another new parameter, \(C_f\) that
related thrust to the nozzle throat area and rocket total pressure:

\[C_f = \frac{T}{P_t A_c}\]

In prose this says:

\begin{quote}
\(C_f\) represents the amount of thrust a rocket can produce given the
stagnation pressure of its propellants and a useful characteristic fluid
area - the choked area of its nozzle.
\end{quote}

Stated differently it is a measure of how effectively we take the
stagnation pressure we generate and turn it into thrust.

And since \(c^*\) is also defined by chamber pressure and nozzle throat
area, \(C_f\) becomes the connection between \(C\) and \(c^*\):

\[C = \frac{T}{\dot{m}} = \frac{C_f P_t A_c}{\dot{m}} = C_f c^*\]

But why split \(C\) into \(C_f\) and \(c^*\) this way?

Remember that \(c^*\) represents the potential of the rocket propellants
themselves to create thrust and essentially depends exclusively on the
thermodynamic characteristics of those propellants. \(C_f\) represents
how well our nozzle can convert the propellant's latent utility into
real thrust and thus depends almost exclusively on the physical nature
of our nozzle. And so as we go about maximizing \(C\), we can divide
that into two separate problems - one of picking propellants (\(c^*\))
and the other of chosing system pressures and nozzle geometry (\(C_f\)).

Since the primary role of the nozzle is to convert gasses into thrust
\(C_f\) can also be seen as a measure of the "goodness" of the nozzle.
It is called \textbf{thrust coefficient}.

\(C_f\) can be expanded from its definition above:

\[C_f = \frac{C}{c^*} = \frac{(P_e - P_0)\frac{A_e}{\dot{m}} + U_e}{\frac{P_t A_c}{\dot{m}}} = \frac{P_e - P_0}{P_t}\frac{A_e}{A_c} + \frac{U_e}{c^*}\]

It is worth noting that, like the thrust equation, there are two pieces
to \(C_f\) - \(\frac{P_e - P_0}{P_t}\frac{A_e}{A_c}\) is representative
of thrust created through pressure force and \(\frac{U_e}{c^*}\) is
representative of the contribution of gas momentum to thrust. We will
refer to these two components when we discuss optimal nozzle expansion
in a minute.

Beyond this things get a little messy and different people attack the
derivation different ways. Rather than putting a whole bunch of alegbra
in here, I'm going to provide you the tools needed to compute \(C_f\)
practically.

\(\frac{U_e}{c^*}\), \(\frac{P_e}{P_t}\) and \(\frac{A_e}{A_c}\) are all
related to the isentropic expansion of gasses through the nozzle to its
exit. The exit mach number, \(M_e\) thus becomes the common parameter
and using classic isentropic relations we can derive the functional
relationship of each with \(M_e\) as the independent variable:

\[\frac{U_e}{c^*} = \frac{\gamma M_e}{\sqrt{1 + \frac{\gamma - 1}{2}M_e^2}}\left[\frac{\gamma + 1}{2}\right]^{\frac{\gamma + 1}{-2(\gamma-1)}}\]

\[\frac{P_e}{P_t} = \left(1 + \frac{\gamma - 1}{2}M_e^2\right)^{\frac{-\gamma}{\gamma-1}}\]

\[\frac{A_e}{A_c} = \left[\frac{\gamma + 1}{2}\right]^{\frac{\gamma + 1}{-2(\gamma - 1)}} \frac{\left(1 + \frac{\gamma-1}{2}M_e^2 \right)^{\frac{\gamma + 1}{2(\gamma - 1)}}}{M_e}\]

This is a useful form because it shows very clearly that
\(\frac{U_e}{c^*}\), \(\frac{P_e}{P_t}\) and \(\frac{A_e}{A_c}\) are not
all independent (they all depend on \(M_e\)). In fact the dimensionality
of this set is one - picking a number for any one of these directlys
sets the others. Furthermore note that these equations, unlike \(c^*\),
have \textbf{no depdencency on stagnation temperature or \(M_w\) and
thus no direct dependency on propellant properties}. They do depend on
\(\gamma\) which is a property of the working fluid but as with \(c^*\),
the dependence is not terribly strong and is really set for us by the
propellant choice we made in optimizing \(c^*\).

I'd like to look at how the parameter we can control directly,
\(\frac{A_e}{A_c}\), affects the others and \(C_f\). Using the equations
above, we will sweep through \(M_e\) and compute the other parameters
directly.

    \subsection{Optimal Nozzle Expansion}\label{optimal-nozzle-expansion}

    \begin{Verbatim}[commandchars=\\\{\}]
{\color{incolor}In [{\color{incolor}25}]:} \PY{n}{gamma} \PY{o}{=} \PY{l+m+mf}{1.4}
         \PY{n}{P0\PYZus{}Pt} \PY{o}{=} \PY{l+m+mf}{0.1}
         \PY{n}{Pe\PYZus{}P0} \PY{o}{=} \PY{l+m+mf}{1.}
         
         \PY{n}{Me} \PY{o}{=} \PY{n}{np}\PY{o}{.}\PY{n}{linspace}\PY{p}{(}\PY{l+m+mf}{1.2}\PY{p}{,} \PY{l+m+mi}{3}\PY{p}{)}
         
         \PY{n}{Ae\PYZus{}Ac} \PY{o}{=} \PY{p}{(}\PY{p}{(}\PY{p}{(}\PY{n}{gamma} \PY{o}{+} \PY{l+m+mf}{1.}\PY{p}{)} \PY{o}{/} \PY{l+m+mf}{2.}\PY{p}{)}\PY{o}{*}\PY{o}{*}\PY{p}{(}\PY{p}{(}\PY{n}{gamma} \PY{o}{+} \PY{l+m+mi}{1}\PY{p}{)}\PY{o}{/}\PY{o}{\PYZhy{}}\PY{l+m+mf}{2.}\PY{o}{/}\PY{p}{(}\PY{n}{gamma} \PY{o}{\PYZhy{}} \PY{l+m+mi}{1}\PY{p}{)}\PY{p}{)} \PY{o}{/} \PY{n}{Me} \PY{o}{*} 
                     \PY{p}{(}\PY{l+m+mf}{1.} \PY{o}{+} \PY{p}{(}\PY{n}{gamma} \PY{o}{\PYZhy{}} \PY{l+m+mi}{1}\PY{p}{)} \PY{o}{/} \PY{l+m+mf}{2.} \PY{o}{*} \PY{n}{Me}\PY{o}{*}\PY{o}{*}\PY{l+m+mi}{2}\PY{p}{)}\PY{o}{*}\PY{o}{*}\PY{p}{(}\PY{p}{(}\PY{n}{gamma} \PY{o}{+} \PY{l+m+mi}{1}\PY{p}{)}\PY{o}{/}\PY{l+m+mf}{2.}\PY{o}{/}\PY{p}{(}\PY{n}{gamma} \PY{o}{\PYZhy{}} \PY{l+m+mi}{1}\PY{p}{)}\PY{p}{)}\PY{p}{)}
         
         \PY{n}{Pe\PYZus{}Pt} \PY{o}{=} \PY{p}{(}\PY{l+m+mf}{1.} \PY{o}{+} \PY{p}{(}\PY{n}{gamma} \PY{o}{\PYZhy{}} \PY{l+m+mf}{1.}\PY{p}{)}\PY{o}{/}\PY{l+m+mf}{2.} \PY{o}{*} \PY{n}{Me}\PY{o}{*}\PY{o}{*}\PY{l+m+mi}{2}\PY{p}{)}\PY{o}{*}\PY{o}{*}\PY{p}{(}\PY{o}{\PYZhy{}}\PY{n}{gamma} \PY{o}{/} \PY{p}{(}\PY{n}{gamma} \PY{o}{\PYZhy{}} \PY{l+m+mi}{1}\PY{p}{)}\PY{p}{)}
         
         \PY{n}{Cf\PYZus{}mom} \PY{o}{=} \PY{p}{(}\PY{n}{gamma} \PY{o}{*} \PY{n}{Me} \PY{o}{/} \PY{n}{np}\PY{o}{.}\PY{n}{sqrt}\PY{p}{(}\PY{l+m+mi}{1} \PY{o}{+} \PY{p}{(}\PY{n}{gamma} \PY{o}{\PYZhy{}} \PY{l+m+mf}{1.}\PY{p}{)} \PY{o}{/} \PY{l+m+mi}{2} \PY{o}{*} \PY{n}{Me}\PY{o}{*}\PY{o}{*}\PY{l+m+mi}{2}\PY{p}{)} \PY{o}{*} 
                     \PY{p}{(}\PY{p}{(}\PY{n}{gamma} \PY{o}{+} \PY{l+m+mi}{1}\PY{p}{)} \PY{o}{/} \PY{l+m+mi}{2}\PY{p}{)}\PY{o}{*}\PY{o}{*}\PY{p}{(}\PY{p}{(}\PY{n}{gamma} \PY{o}{+} \PY{l+m+mf}{1.}\PY{p}{)} \PY{o}{/} \PY{l+m+mi}{2} \PY{o}{/} \PY{p}{(}\PY{l+m+mf}{1.} \PY{o}{\PYZhy{}} \PY{n}{gamma}\PY{p}{)}\PY{p}{)}\PY{p}{)}
         
         \PY{n}{Cf\PYZus{}mom} \PY{o}{=} \PY{p}{(}\PY{n}{np}\PY{o}{.}\PY{n}{sqrt}\PY{p}{(}\PY{l+m+mi}{2} \PY{o}{*} \PY{n}{gamma}\PY{o}{*}\PY{o}{*}\PY{l+m+mi}{2} \PY{o}{/} \PY{p}{(}\PY{n}{gamma} \PY{o}{\PYZhy{}} \PY{l+m+mi}{1}\PY{p}{)} \PY{o}{*} \PY{p}{(}\PY{l+m+mi}{2} \PY{o}{/} \PY{p}{(}\PY{n}{gamma} \PY{o}{+} \PY{l+m+mi}{1}\PY{p}{)}\PY{p}{)}\PY{o}{*}\PY{o}{*}\PY{p}{(}\PY{p}{(}\PY{n}{gamma} \PY{o}{+} \PY{l+m+mi}{1}\PY{p}{)}\PY{o}{/}\PY{p}{(}\PY{n}{gamma} \PY{o}{\PYZhy{}} \PY{l+m+mi}{1}\PY{p}{)}\PY{p}{)} \PY{o}{*} 
                     \PY{p}{(}\PY{l+m+mf}{1.} \PY{o}{\PYZhy{}} \PY{n}{Pe\PYZus{}Pt}\PY{o}{*}\PY{o}{*}\PY{p}{(}\PY{p}{(}\PY{n}{gamma} \PY{o}{\PYZhy{}} \PY{l+m+mi}{1}\PY{p}{)} \PY{o}{/} \PY{n}{gamma}\PY{p}{)}\PY{p}{)}\PY{p}{)}\PY{p}{)}
             
         \PY{n}{Cf\PYZus{}press} \PY{o}{=} \PY{p}{(}\PY{n}{Pe\PYZus{}Pt} \PY{o}{\PYZhy{}} \PY{n}{P0\PYZus{}Pt}\PY{p}{)} \PY{o}{*} \PY{n}{Ae\PYZus{}Ac}
         
         \PY{n}{f} \PY{o}{=} \PY{n}{plt}\PY{o}{.}\PY{n}{figure}\PY{p}{(}\PY{n}{figsize}\PY{o}{=}\PY{p}{(}\PY{l+m+mi}{4}\PY{p}{,}\PY{l+m+mi}{7}\PY{p}{)}\PY{p}{)}
         \PY{n}{plt}\PY{o}{.}\PY{n}{subplot}\PY{p}{(}\PY{l+m+mi}{311}\PY{p}{)}
         \PY{n}{plt}\PY{o}{.}\PY{n}{title}\PY{p}{(}\PY{l+s+sa}{r}\PY{l+s+s1}{\PYZsq{}}\PY{l+s+s1}{\PYZdl{}}\PY{l+s+s1}{\PYZbs{}}\PY{l+s+s1}{frac\PYZob{}P\PYZus{}0\PYZcb{}\PYZob{}P\PYZus{}t\PYZcb{} = }\PY{l+s+si}{\PYZpc{}.1f}\PY{l+s+s1}{\PYZdl{}}\PY{l+s+s1}{\PYZsq{}} \PY{o}{\PYZpc{}} \PY{n}{P0\PYZus{}Pt}\PY{p}{)}
         \PY{n}{plt}\PY{o}{.}\PY{n}{plot}\PY{p}{(}\PY{n}{Pe\PYZus{}Pt} \PY{o}{/} \PY{n}{P0\PYZus{}Pt}\PY{p}{,} \PY{n}{Cf\PYZus{}mom}\PY{p}{,} \PY{l+s+s1}{\PYZsq{}}\PY{l+s+s1}{g\PYZhy{}\PYZhy{}}\PY{l+s+s1}{\PYZsq{}}\PY{p}{,} \PY{n}{label}\PY{o}{=}\PY{l+s+s1}{\PYZsq{}}\PY{l+s+s1}{Cf (momentum)}\PY{l+s+s1}{\PYZsq{}}\PY{p}{)}
         \PY{n}{ax1} \PY{o}{=} \PY{n}{plt}\PY{o}{.}\PY{n}{gca}\PY{p}{(}\PY{p}{)}
         \PY{n}{ax1}\PY{o}{.}\PY{n}{yaxis}\PY{o}{.}\PY{n}{set\PYZus{}major\PYZus{}formatter}\PY{p}{(}\PY{n}{FormatStrFormatter}\PY{p}{(}\PY{l+s+s1}{\PYZsq{}}\PY{l+s+si}{\PYZpc{}.2f}\PY{l+s+s1}{\PYZsq{}}\PY{p}{)}\PY{p}{)}
         \PY{n}{ax1}\PY{o}{.}\PY{n}{set\PYZus{}ylabel}\PY{p}{(}\PY{l+s+sa}{r}\PY{l+s+s1}{\PYZsq{}}\PY{l+s+s1}{\PYZdl{}C\PYZus{}f\PYZdl{} (momentum)}\PY{l+s+s1}{\PYZsq{}}\PY{p}{,} \PY{n}{color}\PY{o}{=}\PY{l+s+s1}{\PYZsq{}}\PY{l+s+s1}{g}\PY{l+s+s1}{\PYZsq{}}\PY{p}{)}
         \PY{n}{ax1}\PY{o}{.}\PY{n}{set\PYZus{}yticks}\PY{p}{(}\PY{n}{np}\PY{o}{.}\PY{n}{linspace}\PY{p}{(}\PY{n}{np}\PY{o}{.}\PY{n}{min}\PY{p}{(}\PY{n}{Cf\PYZus{}mom}\PY{p}{)}\PY{p}{,} \PY{n}{np}\PY{o}{.}\PY{n}{max}\PY{p}{(}\PY{n}{Cf\PYZus{}mom}\PY{p}{)}\PY{p}{,} \PY{l+m+mi}{6}\PY{p}{)}\PY{p}{)}
         \PY{n}{ax1}\PY{o}{.}\PY{n}{set\PYZus{}xticks}\PY{p}{(}\PY{n}{np}\PY{o}{.}\PY{n}{linspace}\PY{p}{(}\PY{l+m+mi}{0}\PY{p}{,} \PY{l+m+mi}{4}\PY{p}{,} \PY{l+m+mi}{5}\PY{p}{)}\PY{p}{)}
         \PY{n}{ax1}\PY{o}{.}\PY{n}{tick\PYZus{}params}\PY{p}{(}\PY{l+s+s1}{\PYZsq{}}\PY{l+s+s1}{y}\PY{l+s+s1}{\PYZsq{}}\PY{p}{,} \PY{n}{colors}\PY{o}{=}\PY{l+s+s1}{\PYZsq{}}\PY{l+s+s1}{g}\PY{l+s+s1}{\PYZsq{}}\PY{p}{)}
         \PY{n}{plt}\PY{o}{.}\PY{n}{axvline}\PY{p}{(}\PY{l+m+mi}{1}\PY{p}{,} \PY{n}{c}\PY{o}{=}\PY{l+s+s1}{\PYZsq{}}\PY{l+s+s1}{r}\PY{l+s+s1}{\PYZsq{}}\PY{p}{,} \PY{n}{ls}\PY{o}{=}\PY{l+s+s1}{\PYZsq{}}\PY{l+s+s1}{:}\PY{l+s+s1}{\PYZsq{}}\PY{p}{)}
         \PY{n}{ax2} \PY{o}{=} \PY{n}{plt}\PY{o}{.}\PY{n}{twinx}\PY{p}{(}\PY{p}{)}
         \PY{n}{ax2}\PY{o}{.}\PY{n}{plot}\PY{p}{(}\PY{n}{Pe\PYZus{}Pt} \PY{o}{/} \PY{n}{P0\PYZus{}Pt}\PY{p}{,} \PY{n}{Cf\PYZus{}press}\PY{p}{,} \PY{l+s+s1}{\PYZsq{}}\PY{l+s+s1}{b\PYZhy{}\PYZhy{}}\PY{l+s+s1}{\PYZsq{}}\PY{p}{)}
         \PY{n}{ax2}\PY{o}{.}\PY{n}{set\PYZus{}yticks}\PY{p}{(}\PY{n}{np}\PY{o}{.}\PY{n}{linspace}\PY{p}{(}\PY{n}{np}\PY{o}{.}\PY{n}{min}\PY{p}{(}\PY{n}{Cf\PYZus{}press}\PY{p}{)}\PY{p}{,} \PY{n}{np}\PY{o}{.}\PY{n}{max}\PY{p}{(}\PY{n}{Cf\PYZus{}press}\PY{p}{)}\PY{p}{,} \PY{l+m+mi}{6}\PY{p}{)}\PY{p}{)}
         \PY{n}{ax2}\PY{o}{.}\PY{n}{yaxis}\PY{o}{.}\PY{n}{set\PYZus{}major\PYZus{}formatter}\PY{p}{(}\PY{n}{FormatStrFormatter}\PY{p}{(}\PY{l+s+s1}{\PYZsq{}}\PY{l+s+si}{\PYZpc{}.1f}\PY{l+s+s1}{\PYZsq{}}\PY{p}{)}\PY{p}{)}
         \PY{n}{ax2}\PY{o}{.}\PY{n}{set\PYZus{}ylabel}\PY{p}{(}\PY{l+s+sa}{r}\PY{l+s+s1}{\PYZsq{}}\PY{l+s+s1}{\PYZdl{}C\PYZus{}f\PYZdl{} (pressure)}\PY{l+s+s1}{\PYZsq{}}\PY{p}{,} \PY{n}{color}\PY{o}{=}\PY{l+s+s1}{\PYZsq{}}\PY{l+s+s1}{b}\PY{l+s+s1}{\PYZsq{}}\PY{p}{)}
         \PY{n}{ax2}\PY{o}{.}\PY{n}{tick\PYZus{}params}\PY{p}{(}\PY{l+s+s1}{\PYZsq{}}\PY{l+s+s1}{y}\PY{l+s+s1}{\PYZsq{}}\PY{p}{,} \PY{n}{colors}\PY{o}{=}\PY{l+s+s1}{\PYZsq{}}\PY{l+s+s1}{b}\PY{l+s+s1}{\PYZsq{}}\PY{p}{)}
         
         \PY{c+c1}{\PYZsh{}f.legend()}
         
         \PY{n}{plt}\PY{o}{.}\PY{n}{subplot}\PY{p}{(}\PY{l+m+mi}{312}\PY{p}{,} \PY{n}{sharex}\PY{o}{=}\PY{n}{ax1}\PY{p}{)}
         \PY{n}{plt}\PY{o}{.}\PY{n}{plot}\PY{p}{(}\PY{n}{Pe\PYZus{}Pt} \PY{o}{/} \PY{n}{P0\PYZus{}Pt}\PY{p}{,} \PY{n}{Cf\PYZus{}mom} \PY{o}{+} \PY{n}{Cf\PYZus{}press}\PY{p}{,} \PY{n}{label}\PY{o}{=}\PY{l+s+s1}{\PYZsq{}}\PY{l+s+s1}{Cf}\PY{l+s+s1}{\PYZsq{}}\PY{p}{)}
         \PY{n}{plt}\PY{o}{.}\PY{n}{axvline}\PY{p}{(}\PY{l+m+mi}{1}\PY{p}{,} \PY{n}{c}\PY{o}{=}\PY{l+s+s1}{\PYZsq{}}\PY{l+s+s1}{r}\PY{l+s+s1}{\PYZsq{}}\PY{p}{,} \PY{n}{ls}\PY{o}{=}\PY{l+s+s1}{\PYZsq{}}\PY{l+s+s1}{:}\PY{l+s+s1}{\PYZsq{}}\PY{p}{)}
         \PY{n}{plt}\PY{o}{.}\PY{n}{ylabel}\PY{p}{(}\PY{l+s+sa}{r}\PY{l+s+s1}{\PYZsq{}}\PY{l+s+s1}{\PYZdl{}C\PYZus{}f\PYZdl{}}\PY{l+s+s1}{\PYZsq{}}\PY{p}{)}
         
         \PY{n}{plt}\PY{o}{.}\PY{n}{subplot}\PY{p}{(}\PY{l+m+mi}{313}\PY{p}{,} \PY{n}{sharex}\PY{o}{=}\PY{n}{ax1}\PY{p}{)}
         \PY{n}{plt}\PY{o}{.}\PY{n}{plot}\PY{p}{(}\PY{n}{Pe\PYZus{}Pt} \PY{o}{/} \PY{n}{P0\PYZus{}Pt}\PY{p}{,} \PY{n}{Ae\PYZus{}Ac}\PY{p}{)}
         \PY{n}{plt}\PY{o}{.}\PY{n}{ylabel}\PY{p}{(}\PY{l+s+sa}{r}\PY{l+s+s1}{\PYZsq{}}\PY{l+s+s1}{\PYZdl{}}\PY{l+s+s1}{\PYZbs{}}\PY{l+s+s1}{frac\PYZob{}A\PYZus{}e\PYZcb{}\PYZob{}A\PYZus{}c\PYZcb{}\PYZdl{}}\PY{l+s+s1}{\PYZsq{}}\PY{p}{)}
         \PY{n}{plt}\PY{o}{.}\PY{n}{xlabel}\PY{p}{(}\PY{l+s+sa}{r}\PY{l+s+s1}{\PYZsq{}}\PY{l+s+s1}{\PYZdl{}}\PY{l+s+s1}{\PYZbs{}}\PY{l+s+s1}{frac\PYZob{}P\PYZus{}e\PYZcb{}\PYZob{}P\PYZus{}0\PYZcb{}\PYZdl{}}\PY{l+s+s1}{\PYZsq{}}\PY{p}{)}
         \PY{n}{fix\PYZus{}fig}\PY{p}{(}\PY{n}{f}\PY{p}{)}
         \PY{n}{f}\PY{o}{.}\PY{n}{savefig}\PY{p}{(}\PY{l+s+s1}{\PYZsq{}}\PY{l+s+s1}{imgs/Cf.pdf}\PY{l+s+s1}{\PYZsq{}}\PY{p}{)}
\end{Verbatim}

    \begin{center}
    \adjustimage{max size={0.9\linewidth}{0.9\paperheight}}{output_6_0.png}
    \end{center}
    { \hspace*{\fill} \\}
    
    Note that \(C_f\) reaches a maximum where \(\frac{P_e}{P_0} = 1\). This
is called optimal expansion. It is a fundamental result in rocket theory
and can be stated:

\begin{quote}
Rocket effective exhaust velocity (and therefore specific impulse) is
maximized when the nozzle expands the exhaust gasses such that they
match the local pressure at the nozzle exit.
\end{quote}

This says that, in general, we want to design a rocket nozzle to match
the exit pressure to ambient pressure. This is difficult to do for a
rocket ascending through the atmosphere where the pressure is
continuosly changing. For traditional nozzles, a compromise that looks
at the average performance over the ascent pressure profile is often
chosen.

And since the way we lower exit pressure is with a bigger expansion
ratio, \(\epsilon = \frac{A_e}{A_c}\), this result explains why upper
stage "vacuum nozzles" are so much bigger than first stage "sea-level"
nozzles as can be seen in graphic above and the side-by-side of the
SpaceX Merlin 1D vacuum (left) and Merlin 1D sea-level (right) engines:

\subsection{Putting it all together}\label{putting-it-all-together}

\textbf{So the separation of \(C\) into \(c^*\) and \(C_f\) separates
concerns between propellant thermodyanmics in \(c^*\) and nozzle
physical parameters in \(C_f\).}

Let's do a quickly compute \(C\), \(c^*\) and \(C_f\) to get a sense of
how we would begin to use them practically. Assume we start with a
plenum full of room-temperature, high pressure nitrogen and then vent it
into vacuum through a nozzle. We'll say our gas thermodynamic properties
are given by:

\begin{itemize}
\tightlist
\item
  \(T_t = 298 \text{K}\)
\item
  \(\gamma = 1.4\)
\item
  \(M = .028 \text{kg/mol}\)
\end{itemize}

Now let's compute \(c^*\), \(C_f\) and then \(C\). For now we will
assume optimal expansion and calculate the associated area ratio. A
useful relation we will use to determine \(M_e\) is:

\[M_e^2 = 2\left(\frac{\frac{Pe}{Pt}^{\frac{1-\gamma}{\gamma}}-1}{\gamma-1}\right) = 2\left(\frac{\left(\frac{P_e}{P_0}\frac{P_0}{P_t}\right)^{\frac{1-\gamma}{\gamma}}-1}{\gamma-1}\right)\]

    \begin{Verbatim}[commandchars=\\\{\}]
{\color{incolor}In [{\color{incolor}99}]:} \PY{n}{gamma} \PY{o}{=} \PY{l+m+mf}{1.4}
         \PY{n}{Ru} \PY{o}{=} \PY{l+m+mf}{8.314}
         \PY{n}{M} \PY{o}{=} \PY{o}{.}\PY{l+m+mo}{02}\PY{l+m+mi}{8}              \PY{c+c1}{\PYZsh{} kg / mol}
         \PY{n}{Tt} \PY{o}{=} \PY{l+m+mi}{298}              \PY{c+c1}{\PYZsh{} K}
         \PY{n}{P0\PYZus{}Pt} \PY{o}{=} \PY{l+m+mf}{0.1}
         \PY{n}{Pe\PYZus{}P0} \PY{o}{=} \PY{l+m+mf}{1.}
         
         \PY{k}{def} \PY{n+nf}{cstar\PYZus{}f}\PY{p}{(}\PY{n}{M}\PY{p}{,} \PY{n}{Tt}\PY{p}{,} \PY{n}{gamma}\PY{p}{)}\PY{p}{:}
             \PY{n}{R} \PY{o}{=} \PY{n}{Ru} \PY{o}{/} \PY{n}{M}
             \PY{k}{return} \PY{n}{np}\PY{o}{.}\PY{n}{sqrt}\PY{p}{(}\PY{n}{R} \PY{o}{*} \PY{n}{Tt} \PY{o}{/} \PY{n}{gamma}\PY{p}{)} \PY{o}{*} \PY{p}{(}\PY{p}{(}\PY{n}{gamma} \PY{o}{+} \PY{l+m+mi}{1}\PY{p}{)} \PY{o}{/} \PY{l+m+mi}{2}\PY{p}{)}\PY{o}{*}\PY{o}{*}\PY{p}{(}\PY{p}{(}\PY{n}{gamma} \PY{o}{+} \PY{l+m+mf}{1.}\PY{p}{)} \PY{o}{/} \PY{l+m+mi}{2} \PY{o}{/} \PY{p}{(}\PY{n}{gamma} \PY{o}{\PYZhy{}} \PY{l+m+mf}{1.}\PY{p}{)}\PY{p}{)}
         
         \PY{k}{def} \PY{n+nf}{Me2\PYZus{}f}\PY{p}{(}\PY{n}{Pe\PYZus{}Pt}\PY{p}{,} \PY{n}{gamma}\PY{p}{)}\PY{p}{:}
             \PY{k}{return} \PY{l+m+mf}{2.} \PY{o}{*} \PY{p}{(}\PY{n}{Pe\PYZus{}Pt}\PY{o}{*}\PY{o}{*}\PY{p}{(}\PY{p}{(}\PY{l+m+mi}{1} \PY{o}{\PYZhy{}} \PY{n}{gamma}\PY{p}{)}\PY{o}{/}\PY{n}{gamma}\PY{p}{)} \PY{o}{\PYZhy{}} \PY{l+m+mi}{1}\PY{p}{)} \PY{o}{/} \PY{p}{(}\PY{n}{gamma} \PY{o}{\PYZhy{}} \PY{l+m+mi}{1}\PY{p}{)}
         
         \PY{k}{def} \PY{n+nf}{Ae\PYZus{}Ac\PYZus{}f}\PY{p}{(}\PY{n}{Me}\PY{p}{,} \PY{n}{gamma}\PY{p}{)}\PY{p}{:}
             \PY{k}{return} \PY{p}{(}\PY{p}{(}\PY{p}{(}\PY{n}{gamma} \PY{o}{+} \PY{l+m+mf}{1.}\PY{p}{)} \PY{o}{/} \PY{l+m+mf}{2.}\PY{p}{)}\PY{o}{*}\PY{o}{*}\PY{p}{(}\PY{p}{(}\PY{n}{gamma} \PY{o}{+} \PY{l+m+mi}{1}\PY{p}{)}\PY{o}{/}\PY{o}{\PYZhy{}}\PY{l+m+mf}{2.}\PY{o}{/}\PY{p}{(}\PY{n}{gamma} \PY{o}{\PYZhy{}} \PY{l+m+mi}{1}\PY{p}{)}\PY{p}{)} \PY{o}{/} \PY{n}{Me} \PY{o}{*} 
                     \PY{p}{(}\PY{l+m+mf}{1.} \PY{o}{+} \PY{p}{(}\PY{n}{gamma} \PY{o}{\PYZhy{}} \PY{l+m+mi}{1}\PY{p}{)} \PY{o}{/} \PY{l+m+mf}{2.} \PY{o}{*} \PY{n}{Me}\PY{o}{*}\PY{o}{*}\PY{l+m+mi}{2}\PY{p}{)}\PY{o}{*}\PY{o}{*}\PY{p}{(}\PY{p}{(}\PY{n}{gamma} \PY{o}{+} \PY{l+m+mi}{1}\PY{p}{)}\PY{o}{/}\PY{l+m+mf}{2.}\PY{o}{/}\PY{p}{(}\PY{n}{gamma} \PY{o}{\PYZhy{}} \PY{l+m+mi}{1}\PY{p}{)}\PY{p}{)}\PY{p}{)}
         
         \PY{k}{def} \PY{n+nf}{Cf\PYZus{}f}\PY{p}{(}\PY{n}{Pe\PYZus{}Pt}\PY{p}{,} \PY{n}{P0\PYZus{}Pt}\PY{p}{,} \PY{n}{Ae\PYZus{}Ac}\PY{p}{,} \PY{n}{Me}\PY{p}{)}\PY{p}{:}
             \PY{k}{return} \PY{p}{(}\PY{p}{(}\PY{n}{Pe\PYZus{}Pt} \PY{o}{\PYZhy{}} \PY{n}{P0\PYZus{}Pt}\PY{p}{)} \PY{o}{*} \PY{n}{Ae\PYZus{}Ac} \PY{o}{+} \PY{p}{(}\PY{n}{gamma} \PY{o}{*} \PY{n}{Me} \PY{o}{/} \PY{n}{np}\PY{o}{.}\PY{n}{sqrt}\PY{p}{(}\PY{l+m+mi}{1} \PY{o}{+} \PY{p}{(}\PY{n}{gamma} \PY{o}{\PYZhy{}} \PY{l+m+mf}{1.}\PY{p}{)} \PY{o}{/} \PY{l+m+mi}{2} \PY{o}{*} \PY{n}{Me}\PY{o}{*}\PY{o}{*}\PY{l+m+mi}{2}\PY{p}{)} \PY{o}{*} 
                     \PY{p}{(}\PY{p}{(}\PY{n}{gamma} \PY{o}{+} \PY{l+m+mi}{1}\PY{p}{)} \PY{o}{/} \PY{l+m+mi}{2}\PY{p}{)}\PY{o}{*}\PY{o}{*}\PY{p}{(}\PY{p}{(}\PY{n}{gamma} \PY{o}{+} \PY{l+m+mf}{1.}\PY{p}{)} \PY{o}{/} \PY{l+m+mi}{2} \PY{o}{/} \PY{p}{(}\PY{l+m+mf}{1.} \PY{o}{\PYZhy{}} \PY{n}{gamma}\PY{p}{)}\PY{p}{)}\PY{p}{)}\PY{p}{)}
         
         \PY{n}{cstar} \PY{o}{=} \PY{n}{cstar\PYZus{}f}\PY{p}{(}\PY{n}{M}\PY{p}{,} \PY{n}{Tt}\PY{p}{,} \PY{n}{gamma}\PY{p}{)}
         \PY{n}{Me} \PY{o}{=} \PY{n}{sqrt}\PY{p}{(}\PY{n}{Me2\PYZus{}f}\PY{p}{(}\PY{n}{Pe\PYZus{}P0} \PY{o}{*} \PY{n}{P0\PYZus{}Pt}\PY{p}{,} \PY{n}{gamma}\PY{p}{)}\PY{p}{)}
         \PY{n}{Ae\PYZus{}Ac} \PY{o}{=} \PY{n}{Ae\PYZus{}Ac\PYZus{}f}\PY{p}{(}\PY{n}{Me}\PY{p}{,} \PY{n}{gamma}\PY{p}{)}
         \PY{n}{Cf} \PY{o}{=} \PY{n}{Cf\PYZus{}f}\PY{p}{(}\PY{n}{Pe\PYZus{}P0} \PY{o}{*} \PY{n}{P0\PYZus{}Pt}\PY{p}{,} \PY{n}{P0\PYZus{}Pt}\PY{p}{,} \PY{n}{Ae\PYZus{}Ac}\PY{p}{,} \PY{n}{Me}\PY{p}{)}
         \PY{n}{C} \PY{o}{=} \PY{n}{cstar} \PY{o}{*} \PY{n}{Cf}
         
         \PY{k}{print} \PY{l+s+s1}{\PYZsq{}}\PY{l+s+s1}{c* = }\PY{l+s+si}{\PYZpc{}.1f}\PY{l+s+s1}{ m/s}\PY{l+s+s1}{\PYZsq{}} \PY{o}{\PYZpc{}} \PY{n}{cstar}
         \PY{k}{print} \PY{l+s+s1}{\PYZsq{}}\PY{l+s+s1}{Me = }\PY{l+s+si}{\PYZpc{}.1f}\PY{l+s+s1}{\PYZsq{}} \PY{o}{\PYZpc{}} \PY{n}{Me}
         \PY{k}{print} \PY{l+s+s1}{\PYZsq{}}\PY{l+s+s1}{Ae\PYZus{}Ac = }\PY{l+s+si}{\PYZpc{}.1f}\PY{l+s+s1}{\PYZsq{}} \PY{o}{\PYZpc{}} \PY{n}{Ae\PYZus{}Ac}
         \PY{k}{print} \PY{l+s+s1}{\PYZsq{}}\PY{l+s+s1}{Cf = }\PY{l+s+si}{\PYZpc{}.2f}\PY{l+s+s1}{\PYZsq{}} \PY{o}{\PYZpc{}} \PY{n}{Cf}
         \PY{k}{print} \PY{l+s+s1}{\PYZsq{}}\PY{l+s+s1}{C = }\PY{l+s+si}{\PYZpc{}.1f}\PY{l+s+s1}{ m/s (Isp = }\PY{l+s+si}{\PYZpc{}.1f}\PY{l+s+s1}{ s)}\PY{l+s+s1}{\PYZsq{}} \PY{o}{\PYZpc{}} \PY{p}{(}\PY{n}{C}\PY{p}{,} \PY{n}{C}\PY{o}{/}\PY{l+m+mf}{9.81}\PY{p}{)}
\end{Verbatim}

    \begin{Verbatim}[commandchars=\\\{\}]
c* = 434.4 m/s
Me = 2.2
Ae\_Ac = 1.9
Cf = 1.26
C = 546.4 m/s (Isp = 55.7 s)

    \end{Verbatim}

    This is pretty unimpressive performance. What if we compute the same for
helium?

    \begin{Verbatim}[commandchars=\\\{\}]
{\color{incolor}In [{\color{incolor}105}]:} \PY{n}{gamma} \PY{o}{=} \PY{l+m+mf}{1.66}
          \PY{n}{M} \PY{o}{=} \PY{o}{.}\PY{l+m+mo}{004}              \PY{c+c1}{\PYZsh{} kg / mol}
          \PY{n}{Tt} \PY{o}{=} \PY{l+m+mi}{298}
          
          \PY{n}{cstar} \PY{o}{=} \PY{n}{cstar\PYZus{}f}\PY{p}{(}\PY{n}{M}\PY{p}{,} \PY{n}{Tt}\PY{p}{,} \PY{n}{gamma}\PY{p}{)}
          \PY{n}{Me} \PY{o}{=} \PY{n}{sqrt}\PY{p}{(}\PY{n}{Me2\PYZus{}f}\PY{p}{(}\PY{n}{Pe\PYZus{}P0} \PY{o}{*} \PY{n}{P0\PYZus{}Pt}\PY{p}{,} \PY{n}{gamma}\PY{p}{)}\PY{p}{)}
          \PY{n}{Ae\PYZus{}Ac} \PY{o}{=} \PY{n}{Ae\PYZus{}Ac\PYZus{}f}\PY{p}{(}\PY{n}{Me}\PY{p}{,} \PY{n}{gamma}\PY{p}{)}
          \PY{n}{Cf} \PY{o}{=} \PY{n}{Cf\PYZus{}f}\PY{p}{(}\PY{n}{Pe\PYZus{}P0} \PY{o}{*} \PY{n}{P0\PYZus{}Pt}\PY{p}{,} \PY{n}{P0\PYZus{}Pt}\PY{p}{,} \PY{n}{Ae\PYZus{}Ac}\PY{p}{,} \PY{n}{Me}\PY{p}{)}
          \PY{n}{C} \PY{o}{=} \PY{n}{cstar} \PY{o}{*} \PY{n}{Cf}
          
          \PY{k}{print} \PY{l+s+s1}{\PYZsq{}}\PY{l+s+s1}{c* = }\PY{l+s+si}{\PYZpc{}.1f}\PY{l+s+s1}{ m/s}\PY{l+s+s1}{\PYZsq{}} \PY{o}{\PYZpc{}} \PY{n}{cstar}
          \PY{k}{print} \PY{l+s+s1}{\PYZsq{}}\PY{l+s+s1}{Me = }\PY{l+s+si}{\PYZpc{}.1f}\PY{l+s+s1}{\PYZsq{}} \PY{o}{\PYZpc{}} \PY{n}{Me}
          \PY{k}{print} \PY{l+s+s1}{\PYZsq{}}\PY{l+s+s1}{Ae\PYZus{}Ac = }\PY{l+s+si}{\PYZpc{}.1f}\PY{l+s+s1}{\PYZsq{}} \PY{o}{\PYZpc{}} \PY{n}{Ae\PYZus{}Ac}
          \PY{k}{print} \PY{l+s+s1}{\PYZsq{}}\PY{l+s+s1}{Cf = }\PY{l+s+si}{\PYZpc{}.2f}\PY{l+s+s1}{\PYZsq{}} \PY{o}{\PYZpc{}} \PY{n}{Cf}
          \PY{k}{print} \PY{l+s+s1}{\PYZsq{}}\PY{l+s+s1}{C = }\PY{l+s+si}{\PYZpc{}.1f}\PY{l+s+s1}{ m/s (Isp = }\PY{l+s+si}{\PYZpc{}.1f}\PY{l+s+s1}{ s)}\PY{l+s+s1}{\PYZsq{}} \PY{o}{\PYZpc{}} \PY{p}{(}\PY{n}{C}\PY{p}{,} \PY{n}{C}\PY{o}{/}\PY{l+m+mf}{9.81}\PY{p}{)}
\end{Verbatim}

    \begin{Verbatim}[commandchars=\\\{\}]
c* = 1085.2 m/s
Me = 2.1
Ae\_Ac = 1.7
Cf = 1.26
C = 1366.9 m/s (Isp = 139.3 s)

    \end{Verbatim}

    And finally hydrogen....

    \begin{Verbatim}[commandchars=\\\{\}]
{\color{incolor}In [{\color{incolor}106}]:} \PY{n}{gamma} \PY{o}{=} \PY{l+m+mf}{1.4}
          \PY{n}{M} \PY{o}{=} \PY{o}{.}\PY{l+m+mo}{002}              \PY{c+c1}{\PYZsh{} kg / mol}
          \PY{n}{Tt} \PY{o}{=} \PY{l+m+mi}{298}
          
          \PY{n}{cstar} \PY{o}{=} \PY{n}{cstar\PYZus{}f}\PY{p}{(}\PY{n}{M}\PY{p}{,} \PY{n}{Tt}\PY{p}{,} \PY{n}{gamma}\PY{p}{)}
          \PY{n}{Me} \PY{o}{=} \PY{n}{sqrt}\PY{p}{(}\PY{n}{Me2\PYZus{}f}\PY{p}{(}\PY{n}{Pe\PYZus{}P0} \PY{o}{*} \PY{n}{P0\PYZus{}Pt}\PY{p}{,} \PY{n}{gamma}\PY{p}{)}\PY{p}{)}
          \PY{n}{Ae\PYZus{}Ac} \PY{o}{=} \PY{n}{Ae\PYZus{}Ac\PYZus{}f}\PY{p}{(}\PY{n}{Me}\PY{p}{,} \PY{n}{gamma}\PY{p}{)}
          \PY{n}{Cf} \PY{o}{=} \PY{n}{Cf\PYZus{}f}\PY{p}{(}\PY{n}{Pe\PYZus{}P0} \PY{o}{*} \PY{n}{P0\PYZus{}Pt}\PY{p}{,} \PY{n}{P0\PYZus{}Pt}\PY{p}{,} \PY{n}{Ae\PYZus{}Ac}\PY{p}{,} \PY{n}{Me}\PY{p}{)}
          \PY{n}{C} \PY{o}{=} \PY{n}{cstar} \PY{o}{*} \PY{n}{Cf}
          
          \PY{k}{print} \PY{l+s+s1}{\PYZsq{}}\PY{l+s+s1}{c* = }\PY{l+s+si}{\PYZpc{}.1f}\PY{l+s+s1}{ m/s}\PY{l+s+s1}{\PYZsq{}} \PY{o}{\PYZpc{}} \PY{n}{cstar}
          \PY{k}{print} \PY{l+s+s1}{\PYZsq{}}\PY{l+s+s1}{Me = }\PY{l+s+si}{\PYZpc{}.1f}\PY{l+s+s1}{\PYZsq{}} \PY{o}{\PYZpc{}} \PY{n}{Me}
          \PY{k}{print} \PY{l+s+s1}{\PYZsq{}}\PY{l+s+s1}{Ae\PYZus{}Ac = }\PY{l+s+si}{\PYZpc{}.1f}\PY{l+s+s1}{\PYZsq{}} \PY{o}{\PYZpc{}} \PY{n}{Ae\PYZus{}Ac}
          \PY{k}{print} \PY{l+s+s1}{\PYZsq{}}\PY{l+s+s1}{Cf = }\PY{l+s+si}{\PYZpc{}.2f}\PY{l+s+s1}{\PYZsq{}} \PY{o}{\PYZpc{}} \PY{n}{Cf}
          \PY{k}{print} \PY{l+s+s1}{\PYZsq{}}\PY{l+s+s1}{C = }\PY{l+s+si}{\PYZpc{}.1f}\PY{l+s+s1}{ m/s (Isp = }\PY{l+s+si}{\PYZpc{}.1f}\PY{l+s+s1}{ s)}\PY{l+s+s1}{\PYZsq{}} \PY{o}{\PYZpc{}} \PY{p}{(}\PY{n}{C}\PY{p}{,} \PY{n}{C}\PY{o}{/}\PY{l+m+mf}{9.81}\PY{p}{)}
\end{Verbatim}

    \begin{Verbatim}[commandchars=\\\{\}]
c* = 1625.5 m/s
Me = 2.2
Ae\_Ac = 1.9
Cf = 1.26
C = 2044.5 m/s (Isp = 208.4 s)

    \end{Verbatim}

    The conclusion from this is that even before we get to chemistry, we can
see how wonderful of a working fluid Hydrogen is due to it's low
molecular weight. Remember this because hydrogen will emerge again and
again as we talk about rockets.

As far as a gas goes, Helium is also pretty good and indeed for cold-gas
propulsion Helium is often a good choice. However, being a noble gas,
helium is useless for its chemical energy (remember \(T_t\) is equally
important to \(M_w\) and so is will never be competetive with the
reacting propellants we will discuss when we get to combustion.

    \subsection{Energy, thermodynamics and
efficiency}\label{energy-thermodynamics-and-efficiency}

So far we have concentrated on propulsion primarily from a conservation
of momentum perspective. But there are a lot interesting observations to
be made when we look at the thermodynamics of propulsion as well.

With rocket propulsion and things in space we are talking about
\textbf{a lot of energy} and to put that in to context, let's see how
much energy our Low Earth Orbit satellite ends up with.

An object moving in a conservative potential field (like Earth's
gravity) has total energy:

\[E = \text{KE} + \text{PE} = \frac{m v^2}{2} + \frac{GMm}{r}\]

or in \textbf{mass-specific energy}

\[\epsilon = \frac{v^2}{2} + \frac{GM}{r}\]

In the case of our satellite in a circular Low Earth Orbit this comes
out to:

\[\epsilon = \frac{v^2}{2} + \frac{\mu}{r_0} - \frac{\mu}{r} = 29.1 \text{MJ/kg} + 4.5 \text{MJ/kg} = \mathbf{33.6} \textbf{MJ/kg}\]

Again note how much of this is due to the orbital velocity of the
rocket. And to put in context just how much energy this is, it is
equivalent to:

\begin{itemize}
\tightlist
\item
  1000x the specific energy of a Jet airliner at 650MPH
\item
  5x the specific chemical energy content of nitroglycerin
\item
  3x the required specific energy to melt aluminum
\end{itemize}

It is no wonder that not much is left when object with this much energy
slams into the atmosphere!

    \subsubsection{Rocket efficiency}\label{rocket-efficiency}

Clearly there is a lot of energy being liberated in rocket engines in
order to put things into space. But how efficient are they?

There are different ways to define energy efficiency, but the the first
we'll look at is the \textbf{thermal efficiency} or how effectively a
rocket takes input energy and converts it to gas kinetic energy:

\[\eta_{thermal} = \frac{\dot{W}_{exhaust}}{P_{in}} = \frac{\dot{m}C^2}{2P_{in}}\]

where \(P_{in}\) is the amount of energy being liberated in time to
power the rocket whether be chemical energy, nuclear or electrical. In
the case of chemical rockets we can define it as the energy released by
the propellants when they combust or:

\[P_{in} = \dot{m} \Delta h^0_{rxn}\]

We saw the following equation in a previous chapter:

\[C = \sqrt{\frac{\gamma R T_0 M_e^2}{1+\frac{\gamma-1}{2}M_e^2}} \left[1 +\frac{1}{\gamma M_e^2}\left(1-\frac{P_0}{P_e}\right)\right]\]

If we assume an optimized nozzle with \(P_e = P_0\) so that the right
part of the equation drops out, and assume \(M_e = 7\):

\[C^2 \sim \frac{59R_uT_0}{5.9M_w} \approx \frac{10 R_u\Delta h^0_{rxn}}{\overline{C_p}M_w}\]

Combining these results and

\[\eta_{thermal} \approx \frac{10R_u}{C_p M_w}\]

If we assume \(M_w = 12 \text{g / mol}\) and
\(\frac{R_u}{\overline{C_p}} \approx \frac{2}{3}\) we get:

\[\eta_{thermal} = \frac{2 \times 10}{3 \times 12} = 56\%\]

This is pretty good for a heat engine! But the efficiency of
accelerating gasses is not really what we care about - we care about the
efficiency of accelerating the vehicle.

    To understand that, we will define another effiency metric, called
\textbf{propulsive efficiency} as such

\[\eta_{prop} = \frac{\dot{W}_{r}}{\dot{W}_{r} + \dot{W}_{exhaust}} = \frac{Tv}{Tv + \frac{1}{2}\dot{m}(C - v)^2}\]

or the fraction of total work that actually goes to accelerating the
vehicle.

\[\frac{Tv}{Tv + \frac{1}{2}\dot{m}(C - v)^2} = \frac{\dot{m}C v}{\dot{m}\left[C v + (C-v)^2\right]}\]

\[\rightarrow \eta_{prop} = 2\frac{\frac{v}{C}}{1+(\frac{v}{C})^2}\]

The overall efficiency is \[\eta = \eta_{thermal}\eta_{prop}\]

A small example demonstrating how \(\eta\) varies over ratio of vehicle
speed to exhaust speed, \(\frac{v}{C}\) is shown below.

    \begin{Verbatim}[commandchars=\\\{\}]
{\color{incolor}In [{\color{incolor}108}]:} \PY{n}{Ru} \PY{o}{=} \PY{l+m+mf}{8.314}        \PY{c+c1}{\PYZsh{} J / mol\PYZhy{}K}
          \PY{n}{Mw} \PY{o}{=} \PY{l+m+mf}{15e\PYZhy{}3}      \PY{c+c1}{\PYZsh{} kg / mol}
          \PY{n}{Cf\PYZus{}max} \PY{o}{=} \PY{l+m+mf}{2.2}
          \PY{n}{Cv0} \PY{o}{=} \PY{l+m+mi}{5} \PY{o}{/} \PY{l+m+mf}{2.} \PY{o}{*} \PY{n}{Ru} \PY{o}{/} \PY{n}{Mw}
          \PY{n}{Cp0} \PY{o}{=} \PY{n}{Cv0} \PY{o}{+}  \PY{n}{Ru} \PY{o}{/} \PY{n}{Mw}
          \PY{n}{k} \PY{o}{=} \PY{n}{Cp0}\PY{o}{/}\PY{n}{Cv0}
          
          \PY{k}{def} \PY{n+nf}{carnot}\PY{p}{(}\PY{n}{Th}\PY{p}{,} \PY{n}{Tc}\PY{p}{)}\PY{p}{:}
              \PY{k}{return} \PY{l+m+mf}{1.} \PY{o}{\PYZhy{}} \PY{n}{Tc} \PY{o}{/} \PY{n}{Th}
          
          \PY{k}{def} \PY{n+nf}{cstar}\PY{p}{(}\PY{n}{Tt}\PY{p}{,} \PY{n}{Mw}\PY{p}{,} \PY{n}{k}\PY{p}{)}\PY{p}{:}
              \PY{k}{return} \PY{n}{np}\PY{o}{.}\PY{n}{sqrt}\PY{p}{(}\PY{n}{k} \PY{o}{*} \PY{n}{Ru} \PY{o}{/} \PY{n}{Mw} \PY{o}{*} \PY{n}{Tt}\PY{p}{)} \PY{o}{/} \PY{n}{k} \PY{o}{/} \PY{n}{np}\PY{o}{.}\PY{n}{sqrt}\PY{p}{(}\PY{p}{(}\PY{l+m+mi}{2} \PY{o}{/} \PY{p}{(}\PY{n}{k}\PY{o}{+}\PY{l+m+mi}{1}\PY{p}{)}\PY{p}{)}\PY{o}{*}\PY{o}{*}\PY{p}{(}\PY{p}{(}\PY{n}{k}\PY{o}{+}\PY{l+m+mi}{1}\PY{p}{)}\PY{o}{/}\PY{n}{k}\PY{o}{\PYZhy{}}\PY{l+m+mi}{1}\PY{p}{)}\PY{p}{)}
          
          \PY{k}{def} \PY{n+nf}{eff\PYZus{}propulsive}\PY{p}{(}\PY{n}{U\PYZus{}C}\PY{p}{)}\PY{p}{:}
              \PY{k}{return} \PY{l+m+mf}{2.} \PY{o}{*} \PY{n}{U\PYZus{}C} \PY{o}{/} \PY{p}{(}\PY{l+m+mf}{1.} \PY{o}{+} \PY{p}{(}\PY{n}{U\PYZus{}C}\PY{p}{)}\PY{o}{*}\PY{o}{*}\PY{l+m+mi}{2}\PY{p}{)}
          
          \PY{n}{T0} \PY{o}{=} \PY{l+m+mf}{300.}
          \PY{n}{Tt} \PY{o}{=} \PY{l+m+mf}{3000.}
          \PY{n}{hg} \PY{o}{=} \PY{n}{Tt} \PY{o}{*} \PY{n}{Cp0}
          
          \PY{n}{Cmax} \PY{o}{=} \PY{n}{Cf\PYZus{}max} \PY{o}{*} \PY{n}{cstar}\PY{p}{(}\PY{n}{Tt}\PY{p}{,} \PY{n}{Mw}\PY{p}{,} \PY{n}{k}\PY{p}{)}
          
          \PY{n}{P\PYZus{}KE} \PY{o}{=} \PY{l+m+mf}{0.5} \PY{o}{*} \PY{n}{Cmax}\PY{o}{*}\PY{o}{*}\PY{l+m+mi}{2}
          \PY{n}{eta\PYZus{}KE} \PY{o}{=} \PY{n}{P\PYZus{}KE} \PY{o}{/} \PY{n}{hg}
          \PY{n}{eta\PYZus{}Carnot} \PY{o}{=} \PY{n}{carnot}\PY{p}{(}\PY{n}{Tt}\PY{p}{,} \PY{n}{T0}\PY{p}{)}
          
          \PY{n}{Ur} \PY{o}{=} \PY{n}{np}\PY{o}{.}\PY{n}{linspace}\PY{p}{(}\PY{l+m+mi}{0}\PY{p}{,} \PY{l+m+mi}{10000}\PY{p}{,} \PY{l+m+mi}{1000}\PY{p}{)}
          
          \PY{n}{Cmax} \PY{o}{=} \PY{n}{Cf\PYZus{}max} \PY{o}{*} \PY{n}{cstar}\PY{p}{(}\PY{n}{Tt}\PY{p}{,} \PY{n}{Mw}\PY{p}{,} \PY{n}{k}\PY{p}{)}
          
          \PY{n}{P\PYZus{}KE} \PY{o}{=} \PY{l+m+mf}{0.5} \PY{o}{*} \PY{n}{Cmax}\PY{o}{*}\PY{o}{*}\PY{l+m+mi}{2}
          \PY{n}{eta\PYZus{}KE} \PY{o}{=} \PY{n}{P\PYZus{}KE} \PY{o}{/} \PY{n}{hg}
          \PY{n}{eta\PYZus{}Carnot} \PY{o}{=} \PY{n}{carnot}\PY{p}{(}\PY{n}{Tt}\PY{p}{,} \PY{n}{T0}\PY{p}{)}
          \PY{n}{eta\PYZus{}propulsive} \PY{o}{=} \PY{n}{eff\PYZus{}propulsive}\PY{p}{(}\PY{n}{Ur} \PY{o}{/} \PY{n}{Cmax}\PY{p}{)}
          
          \PY{c+c1}{\PYZsh{}plt.figure(figsize=(10,5))}
          \PY{n}{plt}\PY{o}{.}\PY{n}{plot}\PY{p}{(}\PY{n}{Ur}\PY{o}{/}\PY{n}{Cmax}\PY{p}{,} \PY{n}{eta\PYZus{}propulsive}\PY{p}{,} \PY{l+s+s1}{\PYZsq{}}\PY{l+s+s1}{b\PYZhy{}}\PY{l+s+s1}{\PYZsq{}}\PY{p}{,} \PY{n}{label}\PY{o}{=}\PY{l+s+s1}{\PYZsq{}}\PY{l+s+s1}{\PYZdl{}}\PY{l+s+s1}{\PYZbs{}}\PY{l+s+s1}{eta\PYZus{}\PYZob{}prop\PYZcb{}\PYZdl{}}\PY{l+s+s1}{\PYZsq{}}\PY{p}{,} \PY{n}{linewidth}\PY{o}{=}\PY{l+m+mi}{3}\PY{p}{)}
          \PY{n}{plt}\PY{o}{.}\PY{n}{plot}\PY{p}{(}\PY{n}{Ur}\PY{o}{/}\PY{n}{Cmax}\PY{p}{,} \PY{n}{eta\PYZus{}propulsive} \PY{o}{*} \PY{n}{eta\PYZus{}KE}\PY{p}{,} \PY{l+s+s1}{\PYZsq{}}\PY{l+s+s1}{r\PYZhy{}}\PY{l+s+s1}{\PYZsq{}}\PY{p}{,} \PY{n}{label}\PY{o}{=}\PY{l+s+s1}{\PYZsq{}}\PY{l+s+s1}{\PYZdl{}}\PY{l+s+s1}{\PYZbs{}}\PY{l+s+s1}{eta\PYZdl{}}\PY{l+s+s1}{\PYZsq{}}\PY{p}{,} \PY{n}{linewidth}\PY{o}{=}\PY{l+m+mi}{3}\PY{p}{)}
          \PY{n}{plt}\PY{o}{.}\PY{n}{plot}\PY{p}{(}\PY{p}{[}\PY{l+m+mf}{1.0}\PY{p}{,} \PY{l+m+mf}{1.0}\PY{p}{]}\PY{p}{,} \PY{p}{[}\PY{l+m+mi}{0}\PY{p}{,} \PY{l+m+mi}{1}\PY{p}{]}\PY{p}{,} \PY{l+s+s1}{\PYZsq{}}\PY{l+s+s1}{m\PYZhy{}\PYZhy{}}\PY{l+s+s1}{\PYZsq{}}\PY{p}{)}
          \PY{n}{plt}\PY{o}{.}\PY{n}{ylim}\PY{p}{(}\PY{l+m+mi}{0}\PY{p}{,}\PY{l+m+mf}{1.1}\PY{p}{)}
          \PY{n}{plt}\PY{o}{.}\PY{n}{xlabel}\PY{p}{(}\PY{l+s+s1}{\PYZsq{}}\PY{l+s+s1}{\PYZdl{}V/C\PYZdl{}}\PY{l+s+s1}{\PYZsq{}}\PY{p}{)}
          \PY{n}{plt}\PY{o}{.}\PY{n}{ylabel}\PY{p}{(}\PY{l+s+s1}{\PYZsq{}}\PY{l+s+s1}{\PYZdl{}}\PY{l+s+s1}{\PYZbs{}}\PY{l+s+s1}{eta\PYZdl{}}\PY{l+s+s1}{\PYZsq{}}\PY{p}{)}
          \PY{n}{plt}\PY{o}{.}\PY{n}{legend}\PY{p}{(}\PY{n}{loc}\PY{o}{=}\PY{l+s+s1}{\PYZsq{}}\PY{l+s+s1}{best}\PY{l+s+s1}{\PYZsq{}}\PY{p}{)}
\end{Verbatim}

\begin{Verbatim}[commandchars=\\\{\}]
{\color{outcolor}Out[{\color{outcolor}108}]:} <matplotlib.legend.Legend at 0x12b537ed0>
\end{Verbatim}
            
    \begin{center}
    \adjustimage{max size={0.9\linewidth}{0.9\paperheight}}{output_17_1.png}
    \end{center}
    { \hspace*{\fill} \\}
    
    And here we see another interesting result much like optimal nozzle
expansion. The rocket is most efficient when the exhaust velocity equals
the vehicle velocity. This is actually quite intuitive - in this
condition the exhaust gasses are left with zero velocity in the static
frame. There is no residual kinetic energy in the exhaust so all of it's
kinetic energy have been transferred to the vehicle. Anywhere else the
exhaust is left with residual kinetic energy which is a loss.

    \subsubsection{Energy losses}\label{energy-losses}

Let's also look at where the inefficiencies are for a launch vehicle
with second stage engine \(I_{sp} = 340\) s and moving at 5,000 m/s.

    \begin{Verbatim}[commandchars=\\\{\}]
{\color{incolor}In [{\color{incolor}110}]:} \PY{k+kn}{from} \PY{n+nn}{matplotlib.sankey} \PY{k+kn}{import} \PY{n}{Sankey}
          
          \PY{n}{f} \PY{o}{=} \PY{n}{plt}\PY{o}{.}\PY{n}{figure}\PY{p}{(}\PY{n}{figsize}\PY{o}{=}\PY{p}{(}\PY{l+m+mi}{10}\PY{p}{,}\PY{l+m+mi}{5}\PY{p}{)}\PY{p}{)}
          \PY{n}{ax} \PY{o}{=} \PY{n}{f}\PY{o}{.}\PY{n}{add\PYZus{}subplot}\PY{p}{(}\PY{l+m+mi}{111}\PY{p}{)}
          \PY{n}{diagram} \PY{o}{=} \PY{n}{Sankey}\PY{p}{(}\PY{n}{ax}\PY{o}{=}\PY{n}{ax}\PY{p}{,} \PY{n}{flows}\PY{o}{=}\PY{p}{[}\PY{l+m+mf}{1.0}\PY{p}{,} \PY{o}{\PYZhy{}}\PY{o}{.}\PY{l+m+mo}{02}\PY{p}{,} \PY{o}{\PYZhy{}}\PY{o}{.}\PY{l+m+mo}{01}\PY{p}{,} \PY{o}{\PYZhy{}}\PY{o}{.}\PY{l+m+mo}{01}\PY{p}{,} \PY{o}{\PYZhy{}}\PY{o}{.}\PY{l+m+mi}{4}\PY{p}{,} \PY{o}{\PYZhy{}}\PY{o}{.}\PY{l+m+mi}{1}\PY{p}{,} \PY{o}{\PYZhy{}}\PY{l+m+mf}{0.46}\PY{p}{]}\PY{p}{,}
                 \PY{n}{labels}\PY{o}{=}\PY{p}{[}\PY{l+s+s1}{\PYZsq{}}\PY{l+s+s1}{Reaction}\PY{l+s+se}{\PYZbs{}n}\PY{l+s+s1}{energy}\PY{l+s+s1}{\PYZsq{}}\PY{p}{,} \PY{l+s+s1}{\PYZsq{}}\PY{l+s+s1}{Incomplete combustion}\PY{l+s+s1}{\PYZsq{}}\PY{p}{,} \PY{l+s+s1}{\PYZsq{}}\PY{l+s+s1}{Heat loss}\PY{l+s+s1}{\PYZsq{}}\PY{p}{,} \PY{l+s+s1}{\PYZsq{}}\PY{l+s+s1}{Friction, etc}\PY{l+s+s1}{\PYZsq{}}\PY{p}{,}
                         \PY{l+s+s1}{\PYZsq{}}\PY{l+s+s1}{Second Law loss }\PY{l+s+se}{\PYZbs{}n}\PY{l+s+s1}{(unavailable thermal energy)}\PY{l+s+s1}{\PYZsq{}}\PY{p}{,} \PY{l+s+s1}{\PYZsq{}}\PY{l+s+s1}{Exhaust kinetic}\PY{l+s+se}{\PYZbs{}n}\PY{l+s+s1}{energy}\PY{l+s+s1}{\PYZsq{}}\PY{p}{,} 
                         \PY{l+s+s1}{\PYZsq{}}\PY{l+s+s1}{Useful}\PY{l+s+s1}{\PYZsq{}}\PY{p}{]}\PY{p}{,}
                 \PY{n}{orientations}\PY{o}{=}\PY{p}{[}\PY{l+m+mi}{0}\PY{p}{,} \PY{l+m+mi}{1}\PY{p}{,} \PY{l+m+mi}{1}\PY{p}{,} \PY{l+m+mi}{1}\PY{p}{,} \PY{l+m+mi}{1}\PY{p}{,} \PY{l+m+mi}{1}\PY{p}{,} \PY{l+m+mi}{0}\PY{p}{]}\PY{p}{,}
                 \PY{n}{trunklength} \PY{o}{=} \PY{l+m+mi}{0}\PY{p}{,} \PY{n}{pathlengths}\PY{o}{=}\PY{p}{[}\PY{l+m+mf}{1.}\PY{p}{,} \PY{l+m+mf}{0.6}\PY{p}{,} \PY{l+m+mf}{0.3}\PY{p}{,} \PY{l+m+mf}{0.1}\PY{p}{,} \PY{l+m+mf}{0.3}\PY{p}{,} \PY{l+m+mf}{0.4}\PY{p}{,} \PY{o}{.}\PY{l+m+mi}{5}\PY{p}{]}\PY{p}{,}
                 \PY{n}{linewidth}\PY{o}{=}\PY{l+m+mi}{2}\PY{p}{)}\PY{o}{.}\PY{n}{finish}\PY{p}{(}\PY{p}{)}
          \PY{n}{diagram}\PY{p}{[}\PY{l+m+mi}{0}\PY{p}{]}\PY{o}{.}\PY{n}{patch}\PY{o}{.}\PY{n}{set\PYZus{}facecolor}\PY{p}{(}\PY{l+s+s1}{\PYZsq{}}\PY{l+s+s1}{\PYZsh{}88FF99}\PY{l+s+s1}{\PYZsq{}}\PY{p}{)}
          \PY{n}{diagram}\PY{p}{[}\PY{l+m+mi}{0}\PY{p}{]}\PY{o}{.}\PY{n}{patch}\PY{o}{.}\PY{n}{set\PYZus{}edgecolor}\PY{p}{(}\PY{l+s+s1}{\PYZsq{}}\PY{l+s+s1}{k}\PY{l+s+s1}{\PYZsq{}}\PY{p}{)}
          \PY{n}{ax}\PY{o}{.}\PY{n}{axis}\PY{p}{(}\PY{l+s+s1}{\PYZsq{}}\PY{l+s+s1}{off}\PY{l+s+s1}{\PYZsq{}}\PY{p}{)}
          \PY{n}{plt}\PY{o}{.}\PY{n}{title}\PY{p}{(}\PY{l+s+s2}{\PYZdq{}}\PY{l+s+s2}{Energy loss diagram}\PY{l+s+s2}{\PYZdq{}}\PY{p}{)}
          \PY{n}{plt}\PY{o}{.}\PY{n}{grid}\PY{p}{(}\PY{n+nb+bp}{False}\PY{p}{)}
\end{Verbatim}

    \begin{center}
    \adjustimage{max size={0.9\linewidth}{0.9\paperheight}}{output_20_0.png}
    \end{center}
    { \hspace*{\fill} \\}
    
    \subsection{Addendum - Specific Fuel
Consumption}\label{addendum---specific-fuel-consumption}

    \subsection{Homework problems.}\label{homework-problems.}

\begin{enumerate}
\def\labelenumi{\arabic{enumi}.}
\tightlist
\item
  Derive c* from isentropic relations.
\item
  Why is the RHS of c* equation difficult to measure for a rocket?
\item
  Compute choked mass flowrate for optimal nozzle example given and a
  1mm diameter nozzle throat.
\end{enumerate}

    \subsection{Appendix - Derivation of U\_e/cstar
todo.}\label{appendix---derivation-of-u_ecstar-todo.}

    \begin{Verbatim}[commandchars=\\\{\}]
{\color{incolor}In [{\color{incolor} }]:} 
\end{Verbatim}


    % Add a bibliography block to the postdoc
    
    
    
    \end{document}
